\documentclass[12pt, titlepage]{article}

\usepackage{booktabs}
\usepackage{tabularx}
\usepackage{hyperref}
\hypersetup{
    colorlinks,
    citecolor=black,
    filecolor=black,
    linkcolor=red,
    urlcolor=blue
}
\usepackage[round]{natbib}
\usepackage{longtable}

%% Comments

\usepackage{color}

\newif\ifcomments\commentstrue %displays comments
%\newif\ifcomments\commentsfalse %so that comments do not display

\ifcomments
\newcommand{\authornote}[3]{\textcolor{#1}{[#3 ---#2]}}
\newcommand{\todo}[1]{\textcolor{red}{[TODO: #1]}}
\else
\newcommand{\authornote}[3]{}
\newcommand{\todo}[1]{}
\fi

\newcommand{\wss}[1]{\authornote{blue}{SS}{#1}} 
\newcommand{\plt}[1]{\authornote{magenta}{TPLT}{#1}} %For explanation of the template
\newcommand{\an}[1]{\authornote{cyan}{Author}{#1}}

%% Common Parts

\newcommand{\progname}{MTOBridge} % PUT YOUR PROGRAM NAME HERE
\newcommand{\authname}{Team 15, Alpha Software Solutions
\\ Badawy, Adham
\\ Yazdinia, Pedram
\\ Jandric, David
\\ Vakili, Farzad
\\ Vezina, Victor
\\ Chiu, Darren} % AUTHOR NAMES                  

\usepackage{hyperref}
    \hypersetup{colorlinks=true, linkcolor=blue, citecolor=blue, filecolor=blue,
                urlcolor=blue, unicode=false}
    \urlstyle{same}


\begin{document}

\title{Verification and Validation Report: \progname} 
\author{\authname}
\date{\today}
	
\maketitle

\pagenumbering{roman}

\section{Revision History}

\begin{tabularx}{\textwidth}{p{2cm}p{3.5cm}X}
\toprule {\bf Date} & {\bf Developer(s)} & {\bf Change}\\
\midrule
March 6 & David & NFR System Testing\\
March 6 & Adham & Design Evaluation\\
March 7 & Pedram & VnV Plan Unambiguous\\
March 7 & Victor & VnV Plan Completeness\\
March 7 & Adham & VnV Plan Overall\\
March 7 & All & FR System Testing\\
March 8 & Victor & SRS Ease of Understanding\\
March 8 & David & SRS Completeness\\
March 8 & Adham & Changes Due to Testing\\
March 8 & All & Unit Testing\\
March 8 & Pedram & SRS Ambiguity\\
March 8 & Darren & VnV Plan Feasibility\\
March 8 & Adham & SRS Testability\\
March 8 & Darren \& Farzad & SRS Abstraction \& Spelling\\
March 8 & Victor & Traceability\\
March 8 & All & Reflection\\
March 8 & Victor & Automated Testing\\
\bottomrule
\end{tabularx}

~\newpage

\section{Symbols, Abbreviations and Acronyms}

Refer to section 1.4 of \href{https://github.com/agentvv/MTOBridge/blob/main/docs/SRS/SRS.pdf}{the SRS}.

\newpage

\tableofcontents

\listoftables %if appropriate

%\listoffigures %if appropriate

\newpage

\pagenumbering{arabic}

\section{Introduction}
The following document is a Verification and Validation Report for the MTOBridge Software. It details the results of carrying out the previously drafted
Verification and Validation Plan. That document can be found \href{https://github.com/agentvv/MTOBridge/blob/main/docs/VnVPlan/VnVPlan.pdf}{here}\\

\section{Document Evaluation}

\subsection{SRS Evaluation}
\subsubsection{Abstraction}
FRs were reviewed and no abstraction issues were found. The fact that the sentences start with "The program should allow the user" helps the author with focusing on what should be provided to the user and not how it would be provided.\\\\
NFRs were reviewed and no significant abstraction issues were found.

\subsubsection{Testability}

For Testability, the FRs are all relatively atomic, with clear fit criteria, they should be easily
testable in combination with the VnV plan.\\

The NFRs are a bit muddier, though many of them also have quantitative fit criteria that make them much more testable. One notable exception is the localization requirement,
while this is testable in theory, in practice we do not have reasonable access to a french translator nor do we have much experience in localization. This could still be a
requirement, but the fit criterion should probably change from measuring the time taken to actually localize the program to a more high level evaluation of the difficulty of
translation. Something like a word count could work, but given a bit more thought it is likely a better fit criterion could be chosen. It would likely be of that general type,
however.\\

\subsubsection{Ambiguity}
In terms of ambiguity, the Functional Requirements correctly reflected the abstract properties of the software with a clear language. Nevertheless, there are changes that can improve the clarity. In the third Functional Requirement, the requirement states the capabilities of the software in visualizing previously specified truck platoons. The wording in this case can be specified to indicate changes can only be made in the truck tab and the final platoon will be an exact copy. We can see a very similar wording with Functional Requirement 7 which is ambiguous in reflecting the source of changes. We also are concerned with Functional Requirements 14 as well as 15 as the definition of "visualizing" can be up to interpretation in these scenarios. We can remove the ambiguity by breaking down the visualization to various components where each would be a separate Functional Requirements as these can often include graphs, charts, animations and more complex objects. 

The Non-Functional Requirements which are meant to describe the qualities of the software are also written in a descriptive manner, nevertheless there is room for improvement. Starting with the very first Requirement, it states that graphics of the software will be informative. This a broad statement which can be ineffective in visualizing the final look. This can be changed to specify simple yet key elements of the UI such as "The graphics will be tabular with a visual element on the right side of the each tab".  Similarly, the second Non-Functional Requirement captures the usability of the software through an intuitive design. This can also be broken down to highlight the intuitive design for each section such as the Truck and the Bridge. This can include familiar design patters or dynamic feedback. Moving on the Performance section, Non-Functional Requirements 10 and 11 aim to evaluate the "slowness" as well as the precision of the program. While there is fit criterion available, there is a need for a further context in each scenario. As for the precision, it is vital to understand the target metrics. Finally, the thirteenth requirement mentions the maintainability aspects of the software. With a standalone windows-based application, there is a guarantee that the program will run the same in any machine. As such, the maintainability would really involve "updates" in this scenario.

\subsubsection{Completeness}
For completeness, we are mainly concerned with the Functional and Nonfunctional requirements, the rest of the document seems fine from this standpoint.\\

As far as the FRs, there seem to be no notable omissions from the 16 stated. Nothing has been exposed through this review, reviews with Dr Yang and her team, or through other
design documents, with one notable exception. The Hazard Analysis identified a need for thread safety in the program, and while implemented in the Rev0 implementation, there
is no mention of this in the FRs. This will need to be added for Rev1.\\

For NFRs, completeness is much harder to judge due to their more nebulous and subjective nature. However, once again no glaring omissions were found by this review.\\

Though this was mentioned in the VnV Plan, it bears repeating that the document is likely not 100 percent complete, but outside of the mentioned thread safety requirements, we
have found it to be close enough to approximate completeness.\\

\subsubsection{Spelling and Grammar}
FRs were reviewed and no significant spelling and grammar errors were found other than small improvements such as using "above-mentioned" instead of "above mentioned" or "long-term" instead of "long term" where a hyphen can be used to join two or more words to form a compound adjective.\\

NFRs were reviewed and no significant spelling and grammar errors were found.

\subsubsection{Ease of Understanding}
The SRS is fairly legible and easy to understand, but it has areas in which it could be improved. The first of these is in the traceability section of the functional and non-functional requirements, which are currently difficult to understand as they are simply the id of the constraint or assumption (e.g., C.1, A.1, etc.). Turning these ids into links which point to the appropriate constraint or assumption would help the reader understand the requirements by allowing them to read the constraint or assumption again if they do not remember them.

Another area in which the SRS lacks legibility is in the figures. None of the figures contain legends which detail what the various symbols and arrows used within the figure represent. This is especially important for the use case diagram (Figure 2), which uses various arrows with different meaning. Adding a simple legend to each figure would solve this issue and improve the readability of the SRS.

\subsection{Design Evaluation}
\subsubsection{Usability} In a meeting with Dr Yang and her team, we collectively agreed upon a score out of 5 for each of the following Usability metrics the program should 
have. Below are the scores decided upon along with a paraphrased summary of the group's reasoning.\\
\begin{itemize}
\item \textbf{Keeps Users informed about its status:} 3/5 for the wrong reasons. Error messages pop up when something goes wrong, but they also tend to crash the program.\\
\item \textbf{Show information in a way users understand:} 4/5. Overall acceptable, but the bridge visualization is not very accurate to what a Civil Engineer would expect\\
\item \textbf{Offer users control:} 2/5. User often loses control of the program due to errors that lead to crashes, and there is no way to undo an error.\\
\item \textbf{Be consistent:} 4/5. Not bad. A couple of the button labels need to be changed to be consistent across tabs.\\
\item \textbf{Prevent errors:} 1/5. The program does basically none of this. Error prevention is almost entirely up to the user. Very bad in its current state.\\
\item \textbf{Have visible information to let users recognize options:} 3/5. The program is fairly simple, and the labels on the buttons imply function well enough to make it
intuitive, but some actual on screen instructions or help hovers would be useful.\\
\item \textbf{Be flexible:} 1/5. There is only really one way to operate the program, and if the user skips any steps to try and speed things up it usually leads to error.\\
\item \textbf{Have no clutter:} 5/5. No discussed issues here, no extraneous information was found on any of the tabs.\\
\item \textbf{Provide plain-language help:} 1/5. No help/documentation tab, and when errors pop up they are C++ jargon and not helpful to a layman user.\\
\end{itemize}
There is quite a lot to improve upon here, and we plan to take this into Rev1 and hopefully remeasure all of these at the end with Yang and her team.

\subsubsection{Flexibility}
After a review of the module guide, stretch goals, and some hands on experience of expanding and extending the Rev0 demo with more features, we came to the conclusion that
the design is quite flexible as is, and not in need of much correction. It is highly modular, and accomplishes low coupling and high cohesion relatively well. 
Changes can be made to the inner workings of one module without really being felt by the other modules. This means that stretch goals such as accepting 2D bridge sketches or
having the UI be customizable should be feasible, or at least not held back by the flexibility and modularity of the design.\\

\subsubsection{Maintainability}
In the interest of time, this review was performed in a more high-level way than originally planned, and no outside volunteers were recruited. With than in mind, 
the overall findings were that while file to file information hiding and separation of concerns were quite good, as described above, there is some cleanup of internal file 
and code structure would help with bug-fixing and maintainability. The issues found were not severe, and many of the classes are relatively simple, but there were examples of 
functions that could be split into a couple of smaller functions to ensure atomicity among functions, not just classes. This is that will be improved for Rev1.\\

\subsubsection{Software Validation Plan}
As mentioned in the VnV plan, this section was a little vague. However, overall everything is going according to plan. The biweekly meetings have kept up consistently, and in
them we present a summary of what we are currently working on and what are goals are, and get feedback on that from Dr Yang, we also discuss anything she or her team wish to
bring up. These meetings have been invaluable to ensure continued alignment between us and them, and will hopefully continue to do so into Rev1. More recently the meetings have
included demonstrations of what we have created for Rev0, and discussions centered around what could be changed or improved to better align with what Dr Yang wants, such as the
Usability section of the Design Verification Plan.\\

\subsection{Verification and Validation Plan Evaluation}

\subsubsection{Completeness}

The primary portion of the V\&V Plan that is not complete is the section on unit testing. While many unit tests have now been written and performed, the V\&V Plan was written at a time when no unit tests were created. The now completed unit tests will be added into the V\&V Plan document so that this section can be fully complete once the project is over. Besides this issue, the V\&V Plan is also missing system tests designed to test NFR.14 and NFR.15. These tests are either going to be added or the NFRs will be reevaluated. Section 2 of the V\&V Plan is also missing a number of abbreviations that are used later in the document.

\subsubsection{Unambiguous}

When evaluating the verification plan in terms of ambiguity, we are mainly concerned with the ability to simply follow the plan as intended with no parts that can be interpreted differently. Overall, each "Verification Plan" section is clearly described with little ambiguity, nevertheless we can make the following changes. Starting with the SRS Verification Plan, part of the completeness verification requires presentation to the client teams at different stages to make sure we are not missing any desired functions. This section can be improved by suggesting a structured set of questions for each presentation that would then help us compare and track our progress much better. As for the Design Verification Plan, defining the aspects of the categories usability, flexibility and maintainability could be better understood by connecting these sections to the actual implementation limitations. Lastly, the Software Validation Plan aims to capture the correctness of the final solution mainly through unstructured meetups and reviews with the client. Although vague by nature, there are ways to improve this step. This can include further qualitative methods such as surveys and public interviews to quantitative methods such as the total number of tests passing. 

\subsubsection{Feasibility}

For input tests, testing every possible input to confirm that the system responds correctly to each one is unreasonable. Instead, inputs are grouped into categories and it is assumed that an input from that category is representative of the whole, such as positive numbers, negative numbers, letters, and symbols for text field inputs.

Tests for Look and Feel and Usability and Humanity stated that surveys or tests would be performed on civil engineers using the program. In practice, the system was verified through meetings with the client and her students who possess a civil engineering background but are not necessarily MTO bridge engineers themselves.

\subsubsection{Overall Objective Accomplishment}
There were three stated objectives outlined in Section 3.2 of the VnV Plan:\\
\begin{itemize}
\item Build confidence that the design documents are complete, verifiable, unambiguous, and generally of high quality\\
\item Build confidence that in important non functional qualities such as Usability, or expose if those qualities aren't there\\
\item Build confidence that the implementation does what we thought it did when we built it\\
\end{itemize}
Reviewing the plan holistically, it seems to accomplish the first two objectives rather well. The section on the document reviews is complete, unambiguous, and feasible,
with the notable exception that the last point of section 4.4 regarding this very review should explicitly mention the objectives of section 3.2 for clarity. Overall though,
no issues. Usability and other non functional qualities are also heavily focused on in sections 4.3, 4.7, and in the non functional tests of Section 5. and finally, section 5
and 6 for System and Unit Testing respectively cover the third objective. The obvious problem here is that as mentioned in the completeness section, the unit testing section
is basically entirely missing in the plan as is. If that is fixed and well filled out, however, the third objective will also be accomplished.\\

\section{System Evaluation}

\subsection{Functional Requirements Evaluation}

\subsubsection{MATLAB}

\paragraph{MATLAB Communication}

\begin{enumerate}

  \item{FR1.ST1\\}

  Control: Manual

  Initial State: None

  Input: None

  Expected Output: MATLAB engine started

  How test will be performed: The test will instantiate our MATLAB engine wrapper.

  Actual Output: MATLAB engine started

  Results Summary: the MATLAB engine starts and commands are ready to be sent to MATLAB.

  \item{FR1.ST2\\}

  Control: Manual

  Initial State: None

  Input: Truck, Bridge, and Analysis configurations

  Expected Output: Any result from standard out or standard error

  How test will be performed: Start the MATLAB engine through our wrapper, create
  some default truck, bridge, and analysis configurations, and try to run the analysis.

  Actual Output: Produces the correct results for the configurations.

  Results Summary: MATLAB returned a result, which was the correct result. It was verified 
  against the corresponding standalone MATLAB calculation.

\end{enumerate}

\subsubsection{Truck Configuration}

\paragraph{Truck Configuration Input}

\begin{enumerate}

  \item{FR2.ST1\\}

  Control: Manual

  Initial State: GUI exists

  Input: Axle load, axle spacing, number of trucks, headway

  Expected Output: GUI fields are filled

  How Test Was Performed: Input fields were manually filled, preview button was hit, and GUI fields were manually checked

  Actual Output: GUI fields were filled

  Results Summary: Pass

  \item{FR2.ST2\\}

  Control: Manual

  Initial State: GUI exists

  Input: Invalid axle load, axle spacing, number of trucks, headway

  Expected Output: GUI fields are filled, highlighted as incorrect

  How Test Was Performed: Input fields were manually filled with invalid values, preview button was hit, and GUI fields were manually checked

  Actual Output: Program Crashed

  Results Summary: Fail, does not behave as expected, and fails to handle invalid input.

\end{enumerate}

\paragraph{Truck Platoon Visualization}

\begin{enumerate}

  \item{FR3.ST1\\}

  Control: Manual

  Initial State: GUI exists

  Input: Valid truck configuration

  Expected Output: A valid visualization of the truck platoon shown

  How Test Was Performed: A valid truck configuration was inputted, the preview button was hit, and the visualization was compared against the oracle.

  Actual Output: A valid visualization of the truck platoon shown

  Results Summary: Pass

  \item{FR3.ST2\\}

  Control: Manual

  Initial State: GUI exists

  Input: Invalid truck configuration

  Expected Output: A message informing the user of an incorrect truck configuration

  How Test Was Performed: An invalid truck configuration was inputted, the preview button was hit, and the result was compared against the oracle.

  Actual Output: Program Crashed

  Results Summary: Fail, does not behave as expected, and fails to handle invalid input.

\end{enumerate}

\paragraph{Save Truck Configuration}

\begin{enumerate}

  \item{FR4.ST1\\}

  Control: Manual

  Initial State: GUI exists

  Input: A truck configuration

  Expected Output: A file with the correctly saved configuration

  How Test Was Performed: Given a truck configuration, the "save configuration" button was hit and a file saved. The contents of the file were compared against the oracle

  Actual Output: A file with the correctly saved configuration

  Results Summary: Pass

\end{enumerate}

\paragraph{Load Truck Configuration}

\begin{enumerate}

  \item{FR5.ST1\\}

  Control: Manual

  Initial State: GUI exists, Truck configuration file exists.

  Input: A truck configuration file

  Expected Output: Truck configuration input fields will be filled correctly

  How Test Was Performed: Given a truck configuration file, the "load configuration" button was hit and the file was loaded. The contents of the input fields were then compared against the oracle.

  Actual Output: Truck configuration input fields was be filled correctly

  Results Summary: Pass

\end{enumerate}

\subsubsection{Bridge Configuration}

\paragraph{Bridge Configuration Input}

\begin{enumerate}

  \item{FR6.ST1\\}

    Control: Manual
            
    Initial State: GUI exists
                    
    Input: Number of spans, span length, discretization length, concerned section
                    
    Expected Output: GUI fields are filled
    
    How Test Was Performed: Input fields were manually filled, preview
    button was hit, and GUI fields were manually checked
            
    Actual Output: GUI fields are filled
    
    Results Summary: Pass
  
  \item{FR6.ST2\\}
  
  Control: Manual
  
  Initial State: GUI exists
  
  Input: Invalid number of spans, bridge length
  
  Expected Output: GUI fields are filled, highlighted as incorrect
  
  How Test Was Performed: Input fields was filled with incorrect information
  
  Actual Output: GUI fields are filled, not highlighted as incorrect
  
  Results Summary: Fail

\end{enumerate}

\paragraph{Bridge Visualization}

\begin{enumerate}
  
  \item{FR7.ST1\\}
  
  Control: Manual
  
  Initial State: GUI exists
  
  Input: valid bridge configuration
  
  Expected Output: A visualization of the effects of their bridge characteristics is displayed. 
  
  How Test Was Performed: User clicks "Preview" button after filling in all the other input fields inside the bridge tab
  
  Actual Output: A visualization of the effects of their bridge characteristics is displayed.
  
  Results Summary: Pass
  
  \item{FR7.ST2\\}
  
  Control: Manual
  
  Initial State: GUI exists
  
  Input: Invalid bridge configuration
  
   Expected Output: A message informing the user of an incorrect bridge configuration
  
  How Test Was Performed: User clicks "Preview" button after filling in all the other input fields inside the bridge tab

  Actual Output: Program Crashes
    
  Results Summary: Fail 
  
\end{enumerate}

\paragraph{Save Bridge Configuration}

\begin{enumerate}

  \item{FR8.ST1\\}
  
  Control: Manual
  
  Initial State: GUI exists
  
  Input: Invalid bridge configuration
  
  Expected Output: A new window pops up asking the user where to save their bridge inputs and the file is saved with the correct format after the location is specified. 
  
  How Test Was Performed: User clicks on the "Save Config" button after filling in all the other input fields inside the bridge tab

  Actual Output: A new window pops up asking the user where to save their bridge inputs and the file is saved with the correct format after the location is specified.
    
  Results Summary: Pass

\end{enumerate}

\paragraph{Load Bridge Configuration}

\begin{enumerate}
  
  \item{FR9.ST1\\}
  
  Control: Manual
  
  Initial State: GUI exists
  
  Input: A bridge configuration File
  
  Expected Output: A new window pops up asking the user to specify which configuration file they want to load. After specifying the file, input fields get filled. 
  
  How Test Was Performed: User clicks on the "Save Config" button after filling in all the other input fields inside the bridge tab

  Actual Output: A new window pops up asking the user to specify which configuration file they want to load. After specifying the file, input fields get filled.
    
  Results Summary: Pass  

\end{enumerate}

\subsubsection{Solver Setup}

\paragraph{Solver Selection}

\begin{enumerate}

  \item{FR10.ST1\\}

  Control: Manual
            
  Initial State: GUI exists
            
  Input: Solver choice of concerned section or critical section made by user
            
  Expected Output: Calculation results are based on the solver choice made by the user
         
  How Test Was Performed: This test was performed by running pre-determined calculations after selecting each of the solver choices and comparing the data output to pre-calculated results, ensuring that the selected solver was being used for calculations. 
	
  Actual Output: Calculation results are based on the solver choice made by the user

  Results Summary: This test passed as the system correctly allowed the user to modify the solver being used for calculations
				
\end{enumerate}

\paragraph{Section of Concern}

\begin{enumerate}

  \item{FR11.ST1\\}

  Control: Manual
            
  Initial State: GUI exists
            
  Input: Concerned section inputted by user via text box
            
  Expected Output: Calculation results are for the concerned section specified by the user
            
  How Test Was Performed: This test was performed by running pre-determined calculations after specifying different concerned sections and comparing the data output to pre-calculated results, ensuring that the specified concerned section was being used for calculations.
					
  Actual Output: Calculation results are for the concerned section specified by the user

  Results Summary: This test passed as the system correctly allowed the user to modify the concerned section being used for calculations

 \end{enumerate}

\paragraph{Discretization Length}

\begin{enumerate}

  \item{FR12.ST1\\}

  Control: Manual
					
  Initial State: GUI exists
					
  Input: Discretization length inputted by user via text box
					
  Expected Output: Calculation results are based on the discretization length specified by the user

  How Test Was Performed: This test was performed by running pre-determined calculations after specifying different discretization lengths and comparing the data output to pre-calculated results, ensuring that the specified discretization length was being used for calculations. 

  Actual Output: Calculation results are based on the discretization length specified by the user

  Results Summary: This test passed as the system correctly allowed the user to modify the discretization length being used for calculations

\end{enumerate}

\paragraph{Force Response}

\begin{enumerate}

\item{FR13.ST1\\}

Control: Manual
					
Initial State: GUI exists
					
Input: Force response choice of positive moment, negative moment, or shear made by user
					
Expected Output: Calculation results are based on the force response choice made by the user

How Test Was Performed: This test was performed by running pre-determined calculations after selecting each of the force response choices and comparing the data output to pre-calculated results, ensuring that the selected force response was being used for calculations.
					
Actual Output: Calculation results are based on the force response choice made by the user

Results Summary: This test passed as the system correctly allowed the user to modify the desired force response used for calculations
				
\end{enumerate}

\subsubsection{Result Visualization}

\paragraph{Concerned Section Result Visualization}

\begin{enumerate}

\item{FR14.ST1\\}

Control: Manual
				
Initial State: GUI exists
					
Input: The user performs calculations using the concerned section solver
					
Expected Output: A visualisation of the calculation results are displayed to the user

How Test Was Performed: This test was performed by running pre-determined concerned section calculations and comparing the visualised output data to pre-calculated results, ensuring that the calculation results are being correctly shown to the user. 
			
Actual Output: A visualisation of the calculation results are displayed to the user

Results Summary: This test passed as the system correctly visualised the results of concerned section calculations
				
\end{enumerate}

\paragraph{Critical Section Result Visualization}

\begin{enumerate}

\item{FR15.ST1\\}

Control: Manual
				
Initial State: GUI exists
					
Input: The user performs calculations using the critical section solver
					
Expected Output: A visualisation of the calculation results are displayed to the user

How Test Was Performed: This test was performed by running pre-determined critical section calculations and comparing the visualised output data to pre-calculated results, ensuring that the calculation results are being correctly shown to the user. 
			
Actual Output: A visualisation of the calculation results are displayed to the user

Results Summary: This test passed as the system correctly visualised the results of critical section calculations
				
\end{enumerate}

\subsubsection{Execution Flow Logging}

\paragraph{Report Generation}

\begin{enumerate}

\item{FR16.ST1\\}

Control: Manual
					
Initial State: Application has been opened with configurations set and able to run analysis
					
Input: Save button pressed and name of file to be saved given as "report.txt" in local directory

How Test Was Performed: Fields were set for bridge, truck, and solver configurations and then an analysis run to generate results. The report save was then tested by clicking the button, defining the file name to save to, and opening the created file to ensure it contained the necessary information.
					
Expected Output: "report.txt" generated containing configurations and analysis results

Actual Output: "report.txt" generated containing configurations and analysis results

Results Summary: Pass
					
\end{enumerate}

\subsection{Nonfunctional Requirements Evaluation}

\subsubsection{Look and Feel Requirements}

\begin{enumerate}

\item{NFR1.ST1\\}

Type: Manual, Static

Initial State: Bridge UI mock ups containing graphic elements will be prepared

Input: Civil engineers/students are presented with a UI mock up.

Expected Output: The engineers/students can correctly identify each UI element physically corresponds to in 90\% of cases.

How Test Was Performed: The developers presented the program to civil engineers/students, and assessed how accurately the engineers could identify the UI elements.

Actual Output: Over 90\% of engineers/students understood almost all of the UI elements were for.

Result Summary: Pass

\end{enumerate}

\subsubsection{Usability and Humanity Requirements}
		
\begin{enumerate}

\item{NFR2.ST1\\}

Type: Manual, Dynamic

Initial State: Application is opened with no past data

Input: Civil engineers/students are presented with the application and a brief explanation

Expected Output: 90\% of the civil engineers/students are able to perform bridge analysis within 5 minutes of introduction

How Test Was Performed: Civil engineers/students were presented with the application and a brief explanation by the developers, then the amount of time it takes them to complete a bridge analysis will be measured.

Actual Output: Over 90\% of the civil engineers/students were able to perform the bridge analysis within 3 minutes of introduction

Results Summary: Pass

\item{NFR3.ST1\\}

Type: Manual, Dynamic

Initial State: Application is opened

Input: Application is running on a computer with 1280x720 display

Expected Output: All UI elements remain visible and accessible

How Test Was Performed: The application was opened on a 1280x720 display and checked to ensure that all UI elements were still visible and accessible.

Actual Output: All UI elements remained visible and accessible

Result Summary: Pass

\item{NFR3.ST2\\}

Repeat test NFR.ST3 for 1366x768 display

Result Summary: Pass

\item{NFR3.ST3\\}

Repeat test NFR.ST3 for 1920x1080 display

Result Summary: Pass

\item{NFR3.ST4\\}

Repeat test NFR.ST3 for 2560x1440 display

Result Summary: Pass

\item{NFR3.ST5\\}

Repeat test NFR.ST3 for 3840x2160 display

Result Summary: Pass

\item{NFR4.ST1\\}

Type: Manual, Dynamic

Initial State: Application is opened with no past data

Input: Font size setting is changed to 8pt

Expected Output: All UI elements remain visible and accessible

How Test Was Performed: The application was set to use 8pt font and opened, then checked to ensure that all UI elements were still visible and accessible

Actual Output: All UI elements remained visible and accessible

\item{NFR4.ST2\\}

Repeat test NFR.ST8 with font size 16pt

Result Summary: Pass

\item{NFR4.ST3\\}

Repeat test NFR.ST8 with font size 24pt

Result Summary: Pass

\item{NFR4.ST4\\}

Repeat test NFR.ST8 with font size 32pt

Result Summary: Pass

\item{NFR5.ST1\\}

Type: Manual, Dynamic

Initial State: Application is not installed

Input: Application is downloaded and installed

Expected Output: Application is ready to use within 30 minutes

How Test Was Performed: The application and MATLAB downloaded (on an internet connection with at least a 10Mbps download speed) and installed, and the application was run. The process was timed to ensure that it takes less than 30 minutes.

Actual Output: Application was ready to use within 15 minutes

Result Summary: Pass

\item{NFR6.ST1\\}

Type: Manual, Static

Initial State: Visually similar UI elements are grouped into categories by developers

Input: Civil engineers/students are asked to associate UI elements with their respective categories

Expected Output: Civil engineers/students sort at least 90\% of UI elements into their predefined categories

How test will be performed: The developers presented the names/functions of UI elements to civil engineers and assessed how accurately the engineers/students could correctly group the UI elements.

Actual Output: Civil engineers/students sorted over 80\% of UI elements into their predefined categories

Result Summary: Fail, concerned section and critical section was categorized by the engineers/students into the analysis tab instead of the bridge tab.

\end{enumerate}

\subsubsection{Performance Requirements}

\begin{enumerate}

\item{NFR7.ST1\\}

Type: Automatic, Dynamic

Initial State: Application is opened with no past data

Input: The application is run with invalid inputs (zeroes, negative numbers, strings, etc.)

Expected Output: The application does not freeze or crash

How Test Was Performed: The application was automatically run with many different invalid inputs, and checked that the application does not freeze or crash

Actual Output: The application crashed with many different invalid inputs

Results Summary: Fail, the application could not reliably handle invalid inputs, crashing with almost all tested invalid inputs

\item{NFR8.ST1\\}

Type: Manual, Dynamic

Initial State: Application is installed but the MATLAB files are not installed

Input: The application is opened

Expected Output: The application will display an error regarding the missing MATLAB files

How Test Was Performed: The application was manually installed without the required MATLAB files, and was then opened

Actual Output: The application displayed an error message regarding the missing MATLAB files

Results Summary: Pass, the system was able to give an error message in this situation.

\item{NFR9.ST1\\}

Type: Automatic, Dynamic

Initial State: Application is opened with no past data

Input: Various UI elements are interacted with

Expected Output: The UI elements react to the interaction within 100ms

How Test Was Performed: Various interactions with UI elements were simulated on a computer with hardware similar to that of an MTO engineer's, and the speed of the UI response was measured automatically

Actual Output: The UI elements reacted well within 100ms, approaching almost instantaneous.

Results Summary: Pass, the system was very responsive, and the UI elements responded immediately to input

\item{NFR10.ST1\\}

Type: Automatic, Dynamic

Initial State: Application is opened with no past data

Input: Various valid inputs

Expected Output: Total execution time of calculations will not exceed underlying MATLAB script's execution time by more than 10\%

How Test Was Performed: Many different analyses were performed and timed on a computer with hardware similar to that of an MTO engineer's.

Actual Output: The execution time of calculations was at most ~5\% of the underlying MATLAB script's execution time.

Results Summary: Pass

\end{enumerate}

\subsubsection{Maintainability and Support Requirements}

\begin{enumerate}

\item{NFR13.ST1\\}

Type: Manual, Static

Initial State: The program code is contained in multiple files

Input: Measure code file length in lines

Expected Output: At least 75\% of files will contain 750 lines of code or less

How Test Was Performed: File length was checked manually using Visual Studio

Actual Output/Results: Over 75\% of files contain $<$750 lines of code

Result Summary: Pass

\item{NFR13.ST2\\}

Type: Manual, Static

Initial State: The program code is contained in multiple files

Input: Measure method length in lines

Expected Output: At least 75\% of methods will contain 75 lines of code or less

How Test Was Performed: Method length was checked manually using Visual Studio

Actual Output: Over 75\% of methods contain $<$75 lines of code

Result Summary: Pass

\item{NFR13.ST3\\}

Type: Automatic, Static

Initial State: The program code is contained in multiple files

Input: Measure length of code lines in characters

Expected Output: At least 75\% of lines will contain 120 characters or less

How Test Was Performed: The clang-tidy linter was used to automatically measure line length

Actual Output: Over 75\% of lines contain $<$120 characters

Result Summary: Pass

\item{NFR13.ST4\\}

Type: Automatic, Static

Initial State: The program code is contained in multiple files

Input: Measure nesting depth of methods

Expected Output: At least 75\% of methods will have a maximum nesting depth of 5 or less

How Test Was Performed: The clang-tidy linter was used to automatically measure nesting depth

Actual Output: Over 75\% of methods have a nesting depth of 5 or less

Result Summary: Pass

\end{enumerate}

\section{Unit Testing}

\subsection{MATLAB}

\begin{enumerate}
  
  \item {CalculationCaller.UT1}

  Control: Automatic

  Method: std::shared\_ptr$<$matlab::cpplib::MATLABLibrary$>$ getMatlab()

  Input: None

  Expected Output: MATLAB pointer exists and is valid

  Actual Output: MATLAB pointer exists and is valid

  Results Summary: Pass

  \item {CalculationCaller.UT2}

  Control: Automatic

  Method: CalculationOutputT runCalculation(CalculationInputT)

  Input: The default truck, bridge, and solver configuration

  Expected Output: The output is valid, with correct first axle position,
  equal number of axle positions and forces, and critical section and force undefined.

  Actual Output: Incorrect critical section.

  Results Summary: Fail, the critical section is uninitialized, instead of being
  an impossible value (-1)

  \item {CalculationCaller.UT3}

  Control: Automatic

  Method: CalculationOutputT runCalculation(CalculationInputT)

  Input: Invalid Axle Load

  Expected Output: An empty CalculationOutputT

  Actual Output: Exception, incorrect number of array indices

  Results Summary: Fail, the invalid axle load sequence is not caught, causing an uncaught exception.

  \item {CalculationCaller.UT4}

  Control: Automatic

  Method: CalculationOutputT runCalculation(CalculationInputT)

  Input: Invalid Axle Spacing

  Expected Output: An empty CalculationOutputT

  Actual Output: Exception, incorrect number of array indices

  Results Summary: Fail, the invalid axle spacing sequence is not caught, causing an uncaught exception.
  
  \item {CalculationCaller.UT5}

  Control: Automatic

  Method: CalculationOutputT runCalculation(CalculationInputT)

  Input: Negative Number of Trucks

  Expected Output: An empty CalculationOutputT

  Actual Output: Exception, array indices must be positive integers or logical values

  Results Summary: Fail, the invalid number of trucks is not caught, causing an uncaught exception.

  \item {CalculationCaller.UT6}

  Control: Automatic

  Method: CalculationOutputT runCalculation(CalculationInputT)

  Input: Negative Headway

  Expected Output: An empty CalculationOutputT

  Actual Output: Debug assertion, out of bounds access violation

  Results Summary: Fail, the invalid headway is not caught, causing potential catastrophic failure.

  \item {CalculationCaller.UT7}

  Control: Automatic

  Method: CalculationOutputT runCalculation(CalculationInputT)

  Input: Invalid Number of Bridge Spans

  Expected Output: An empty CalculationOutputT

  Actual Output: Debug assertion, out of bounds access violation, or segmentation fault

  Results Summary: Fail, the invalid number of spans is not caught, causing catastrophic failure.

  \item {CalculationCaller.UT8}

  Control: Automatic

  Method: CalculationOutputT runCalculation(CalculationInputT)

  Input: Invalid Span Length

  Expected Output: An empty CalculationOutputT

  Actual Output: Exception, array indices must be positive integers or logical values

  Results Summary: Fail, the invalid span lengths not caught, causing an uncaught exception.

  \item {CalculationCaller.UT9}

  Control: Automatic

  Method: CalculationOutputT runCalculation(CalculationInputT)

  Input: Negative Concerned Section

  Expected Output: An empty CalculationOutputT

  Actual Output: Command properly runs and returns some output

  Results Summary: Fail, the invalid concerned section not caught, causing incorrect output.

  \item {CalculationCaller.UT10}

  Control: Automatic

  Method: CalculationOutputT runCalculation(CalculationInputT)

  Input: Invalid Discretization Length

  Expected Output: An empty CalculationOutputT

  Actual Output: Command properly runs and returns some output

  Results Summary: Fail, the invalid concerned section not caught, causing incorrect output.

  \item {Engine.UT1}

  Control: Automatic

  Method: Engine getInstance()

  Input: None

  Expected Output: Engine starts and initializes MATLAB

  Actual Output: Engine starts and initializes MATLAB

  Results Summary: Pass

  \item {Engine.UT2}

  Control: Automatic

  Method: QThread thread()

  Input: None

  Expected Output: Main thread and Engine thread are different

  Actual Output: Main thread and Engine thread are different

  Results Summary: Pass

  \item {Engine.UT3}

  Control: Automatic

  Method: CalculationOutputT runCommand(CalculationInputT)

  Input: The default truck, bridge, and solver configuration

  Expected Output: Engine finishes running the command without exception

  Actual Output: Engine finishes running the command without exception

  Results Summary: Pass

  \item {Engine.UT4}

  Control: Automatic

  Method: void stopEngine()

  Input: The default truck, bridge, and solver configuration

  Expected Output: Engine kills MATLAB and it's own thread

  Actual Output: Engine kills MATLAB and it's own thread

  Results Summary: Pass

\end{enumerate}

\subsection{Bridge Configuration}
\begin{enumerate}
    \item{BridgeConfiguration.UT1\\}
    
    Control: Automatic
    
    Method:  void updateNumberOfSpans(QString newNumberOfSpans)
    
    Input: QString::fromStdString("3")
    
    Expected Output: NumberofSpans changes to 3

    Actual Output: NumberofSpans changes to 3

    Results Summary: Pass

    \item{BridgeConfiguration.UT2\\}
    
    Control: Automatic
    
    Method:  void updateNumberOfSpans(QString newNumberOfSpans)
    
    Input: QString::fromStdString("5")
    
    Expected Output: InvalidConfiguration Exception

    Actual Output: NumberofSpans changes to 5

    Results Summary: Fail

    \item {BridgeConfiguration.UT3\\}
    
    Control: Automatic
    
    Method:  void updateNumberOfSpans(QString newNumberOfSpans)
    
    Input: QString::fromStdString("number")
    
    Expected Output: InvalidConfiguration Exception

    Actual Output: NumberofSpans changes to 0

    Results Summary: Fail
    
    \item {BridgeConfiguration.UT4\\}
    
    Control: Automatic
    
    Method:  void updateConcernedSection(QString newConcernedSection)
    
    Input: QString::fromStdString("15")
    
    Expected Output: ConcernedSection changes to 15

    Actual Output: NumberofSpans changes to 15

    Results Summary: Pass

    \item  {BridgeConfiguration.UT5\\}
    
    Control: Automatic
    
    Method:  void updateConcernedSection(QString newConcernedSection)
    
    Input: QString::fromStdString("NUMBER")
    
    Expected Output: InvalidConfiguration Exception

    Actual Output: NumberofSpans changes to 0

    Results Summary: Fail

    \item  {BridgeConfiguration.UT6\\}
    
    Control: Automatic
    
    Method:  void updateDiscretizationLength(QString newDiscretizationLength)
    
    Input: QString::fromStdString("0.5")
    
    Expected Output: DiscretizationLength changes to 0.5

    Actual Output: DiscretizationLength changes to 0.5

    Results Summary: Pass

    \item {BridgeConfiguration.UT7\\}

    Control: Automatic
    
    Method:  void updateDiscretizationLength(QString newDiscretizationLength)
    
    Input: QString::fromStdString("NUMBER")
    
    Expected Output: InvalidConfiguration Exception

    Actual Output: DiscretizationLength changes to 0

    Results Summary: Fail

    \item {BridgeConfiguration.UT8\\}

    Control: Automatic
    
    Method:  void updateSpanLength (QString newNumberOfSpans)
    
    Input: QString::fromStdString("20 30")
    
    Expected Output: SpanLength changes to {20, 30}

    Actual Output: SpanLength changes to {20, 30}

    Results Summary: Pass

    \item {BridgeConfiguration.UT9\\}
    Control: Automatic
    
    Method:  void updateSpanLength (QString newNumberOfSpans)
    
    Input: QString::fromStdString("20,30")
    
    Expected Output: SpanLength changes to {20, 30}

    Actual Output: SpanLength changes to {20}

    Results Summary: Fail

    \item {BridgeConfiguration.UT10\\}
    Control: Automatic
    
    Method:  void updateSpanLength (QString newNumberOfSpans)
    
    Input: QString::fromStdString("number1 number2")
    
    Expected Output: InvalidConfiguration Exception

    Actual Output: Program Crashes

    Results Summary: Fail

    \item {BridgeConfiguration.UT11\\}

    Control: Automatic
    
    Method:  vector$<$double$>$ convertQstringToDoubleVector(QString qstring)
    
    Input: QString::fromStdString("30.5 40")
    
    Expected Output: {30.5, 40}

    Actual Output:{30.5, 40}

    Results Summary: Pass

    \item {BridgeConfiguration.UT12\\\\}
    Control: Automatic
    
    Method:  vector$<$double$>$ convertQstringToDoubleVector(QString qstring)
    
    Input: QString::fromStdString("20,30")
    
    Expected Output: {20, 30}

    Actual Output: {20}

    Results Summary: Fail

    \item {BridgeConfiguration.UT13\\\\}
    Control: Automatic
    
    Method:  vector$<$double$>$ convertQstringToDoubleVector(QString qstring)
    
    Input: QString::fromStdString("number1 number2")
    
    Expected Output: InvalidConfiguration Exception

    Actual Output: Program Crashes

    Results Summary: Fail
\end{enumerate}
\subsection{Platoon Configuration}
\begin{enumerate}
\item{PlatoonConfiguration.UT1}\\\\
  Control: Automatic

  Method: UpdateAxleLoad()

  Input: QString "53.4 75.6 75.6 75.6 75.6"

  Expected Output: Platoon Configuration AxleLoads field should be a list that equals the input

  Actual Output: Platoon Configuration AxleLoads field was a list that equalled the input

  Results Summary: Pass
\item{PlatoonConfiguration.UT2}\\\\
  Control: Automatic

  Method: UpdateAxleLoad()

  Input: QString "53.4,75.6,75.6,75.6,75.6"

  Expected Output: Platoon Configuration AxleLoads field should be a list that equals the input

  Actual Output: Platoon Configuration AxleLoads field was a mess that did not match input, but no exception thrown

  Results Summary: Fail, needs to be able to process comma separated lists as well as space separated lists.

\item{PlatoonConfiguration.UT3}\\\\
  Control: Automatic

  Method: UpdateAxleLoad()

  Input: QString "I like turtles"

  Expected Output: Function should return and no changes should be made to PlatoonConfiguration AxleLoads field

  Actual Output: Exception thrown.

  Results Summary: Fail, failed to handle invalid input.

\item{PlatoonConfiguration.UT4}\\\\
  Control: Automatic

  Method: UpdateAxleSpacing()

  Input: QString "3.6576 1.2192 9.4488 1.2192"

  Expected Output: Platoon Configuration AxleSpacings field should be a list that equals the input

  Actual Output: Platoon Configuration AxleSpacings field was a list that equalled the input

  Results Summary: Pass.
\item{PlatoonConfiguration.UT5}\\\\
  Control: Automatic

  Method: UpdateAxleSpacing()

  Input: QString "3.6576,1.2192,9.4488,1.2192"

  Expected Output: Platoon Configuration AxleSpacings field should be a list that equals the input

  Actual Output: Platoon Configuration AxleSpacings field was a mess that did not match input, but no exception thrown

  Results Summary: Fail, needs to be able to process comma separated lists as well as space separated lists.
  \item{PlatoonConfiguration.UT6}\\\\
  Control: Automatic

  Method: UpdateAxleSpacing()

  Input: QString "I hate turtles"

  Expected Output: Function should return and no changes should be made to PlatoonConfiguration AxleSpacings field

  Actual Output: Exception thrown.

  Results Summary: Fail, failed to handle invalid input.
  \item{PlatoonConfiguration.UT7}\label{PlatoonConfiguration.UT7}\\
  
  Control: Automatic

  Method: UpdateNumberOfTrucks()

  Input: QString "15"

  Expected Output: Platoon Configuration NumTrucks field should be 15

  Actual Output: Platoon Configuration NumTrucks field was 15

  Results Summary: Pass.
\item{PlatoonConfiguration.UT8}\\\\
  Control: Automatic

  Method: UpdateNumberOfTrucks()

  Input: QString "-8"

  Expected Output: Function should return and no changes should be made to PlatoonConfiguration NumTrucks field

  Actual Output: Platoon Configuration NumTrucks field was -8

  Results Summary: Fail, should not allow a negative number of trucks.
  \item{PlatoonConfiguration.UT9}\\\\
  Control: Automatic

  Method: UpdateNumberOfTrucks()

  Input: QString "Booyakasha"

  Expected Output: Function should return and no changes should be made to PlatoonConfiguration NumTrucks field

  Actual Output: Exception thrown.

  Results Summary: Fail, failed to handle invalid input.
  \item{PlatoonConfiguration.UT10}\\\\
  Control: Automatic

  Method: UpdateHeadway()

  Input: QString "10"

  Expected Output: Platoon Configuration Headway field should be 10

  Actual Output: Platoon Configuration Headway field was 10

  Results Summary: Pass.
\item{PlatoonConfiguration.UT11}\\\\
  Control: Automatic

  Method: UpdateHeadway()

  Input: QString "-5"

  Expected Output: Function should return and no changes should be made to PlatoonConfiguration Headway field

  Actual Output: Platoon Configuration Headway field was -5

  Results Summary: Fail, should not allow a negative number of trucks.
  \item{PlatoonConfiguration.UT12}\\\\
  Control: Automatic

  Method: UpdateHeadway()

  Input: QString "Three minutes of playtime"

  Expected Output: Function should return and no changes should be made to PlatoonConfiguration Headway field

  Actual Output: Exception thrown.

  Results Summary: Fail, failed to handle invalid input.
\end{enumerate}
\subsection{Truck Configuration}

\subsection{Saver}
\begin{enumerate}
    \item{Saver.UT1\\}
    
    Control: Automatic
    
    Method:  void savePlatoonConfiguration(TruckT Platoon)
    
    Input: TruckT Platoon = \{.axleLoad = \{53.4, 75.6, 75.6, 75.6, 75.6\},
                        .axleSpacing = \{3.6576, 1.2192, 9.4488, 1.2192\},
                        .numberOfTrucks = 3,
                        .headway = 5\}
    
    Expected Output: .trk file created at the path 

    Actual Output: .trk file created at the path

    Results Summary: Pass

    \item{Saver.UT2\\}
    
    Control: Automatic
    
    Method:  void savePlatoonConfiguration(TruckT Platoon)
    
    Input: TruckT Platoon = \{.axleLoad = \{53.4, 75.6, 75.6, 75.6, 75.6\},
                        .axleSpacing = \{\},
                        .numberOfTrucks = 3,
                        .headway = 5\}
    
    Expected Output: Warning Message due to the empty axleSpacing

    Actual Output: Application Crash

    Results Summary: Fail

    \item{Saver.UT3\\}
    
    Control: Automatic
    
    Method:  void saveBridgeConfiguration(BridgeT Bridge)
    
    Input: BridgeT Bridge = \{.numberSpans = 2,
                         .spanLength = \{20, 20\},
                         .concernedSection = 10,
                         .discretizationLength = 0.1\}
    
    Expected Output: .brg file created at the path

    Actual Output: .brg file created at the path

    Results Summary: Pass

    \item{Saver.UT4\\}
    
    Control: Automatic
    
    Method:  void saveBridgeConfiguration(BridgeT Bridge)
    
    Input: BridgeT Bridge = \{.numberSpans = 2,
                         .spanLength = \{\},
                         .concernedSection = 10,
                         .discretizationLength = 0.1\}
    
    Expected Output: Warning Message due to the empty spanLength

    Actual Output: Application Crash

    Results Summary: Fail

    \item{Saver.UT5\\}
    
    Control: Automatic
    
    Method:  void saveSolverConfiguration(SolverT Solver)
    
    Input: SolverT Solver = \{solverConfig = \{.forceType = mtobridge::MockSolverT::POSITIVE\_MOMENT,
                         .solverType = mtobridge::MockSolverT::CONCERNED\}
    
    Expected Output: .slv file created at the path

    Actual Output: .slv file created at the path

    Results Summary: Pass

    \item{Saver.UT6\\}
    
    Control: Automatic
    
    Method:  void saveSolverConfiguration(SolverT Solver)
    
    Input: SolverT Solver = \{solverConfig = \{.forceType = NULL,
                         .solverType = mtobridge::MockSolverT::CONCERNED\}
    
    Expected Output: Warning Message due to the empty forceType

    Actual Output: Application Crash

    Results Summary: Fail
    
\end{enumerate}

    \subsection{Loader}
\begin{enumerate}
    \item{Loader.UT1\\}
    
    Control: Automatic
    
    Method:  void loadPlatoonConfiguration()
    
    Input: String Platoon = \{.axleLoad = \{53.4, 75.6, 75.6, 75.6, 75.6\},
                        .axleSpacing = \{3.6576, 1.2192, 9.4488, 1.2192\},
                        .numberOfTrucks = 3,
                        .headway = 5\}
    
    Expected Output: Truck Object File returned 

    Actual Output: Truck Object File returned
    
    Results Summary: Pass

    \item{Loader.UT2\\}
    
    Control: Automatic
    
    Method:  void loadPlatoonConfiguration()
    
    Input: String Platoon = \{.axleLoad = \{53.4, 75.6, 75.6, 75.6, 75.6\},
                        .axleSpacing = \{\},
                        .numberOfTrucks = 3,
                        .headway = 5\}
    
    Expected Output: Warning Message due to the empty axleSpacing

    Actual Output: Application Crash

    Results Summary: Fail

    \item{Loader.UT3\\}
    
    Control: Automatic
    
    Method:  void loadBridgeConfiguration()
    
    Input: String Bridge = \{.numberSpans = 2,
                         .spanLength = \{20, 20\},
                         .concernedSection = 10,
                         .discretizationLength = 0.1\}
    
    Expected Output: Bridge Object File returned 

    Actual Output: Bridge Object File returned 

    Results Summary: Pass

    \item{Loader.UT4\\}
    
    Control: Automatic
    
    Method:  void loadBridgeConfiguration()
    
    Input: String Bridge = \{.numberSpans = 2,
                         .spanLength = \{\},
                         .concernedSection = 10,
                         .discretizationLength = 0.1\}
    
    Expected Output: Warning Message due to the empty spanLength

    Actual Output: Application Crash

    Results Summary: Fail

    \item{Loader.UT5\\}
    
    Control: Automatic
    
    Method:  void loadSolverConfiguration()
    
    Input: String Solver = \{solverConfig = \{.forceType = mtobridge::MockSolverT::POSITIVE\_MOMENT,
                         .solverType = mtobridge::MockSolverT::CONCERNED\}
    
    Expected Output: Solver Object File returned 

    Actual Output: Solver Object File returned 

    Results Summary: Pass

    \item{Loader.UT6\\}
    
    Control: Automatic
    
    Method:  void loadSolverConfiguration()
    
    Input: String Solver = \{solverConfig = \{.forceType = NULL,
                         .solverType = mtobridge::MockSolverT::CONCERNED\}
    
    Expected Output: Warning Message due to the empty forceType

    Actual Output: Application Crash

    Results Summary: Fail

    
\end{enumerate}

\subsection{Solver Configuration}

\begin{enumerate}

    \item{Solver.UT1\\}
    
    Control: Automatic

    Initial State: Solver.force == POSITIVE\_MOMENT
    
    Method: static std::string getForceType()
    
    Input: None
    
    Expected Output: "Positive Moment"

    Actual Output: "Positive Moment"

    Results Summary: Pass

    \item{Solver.UT2\\}
    
    Control: Automatic

    Initial State: Solver.solver == CONCERNED
    
    Method: static std::string getSolverType()
    
    Input: None
    
    Expected Output: "Concerned Section"

    Actual Output: "Concerned Section"

    Results Summary: Pass

    \item{Solver.UT3\\}
    
    Control: Automatic

    Initial State: Solver.force == POSITIVE\_MOMENT
    
    Method: static void updateForceType(std::string forceStr)
    
    Input: forceStr = "Negative Moment"
    
    Expected Output: Solver.force ==  NEGATIVE\_MOMENT

    Actual Output: Solver.force == NEGATIVE\_MOMENT

    Results Summary: Pass

    \item{Solver.UT4\\}
    
    Control: Automatic

    Initial State: Solver.force == NEGATIVE\_MOMENT
    
    Method: static void updateForceType(std::string forceStr)
    
    Input: forceStr = "Shear"
    
    Expected Output: Solver.force == SHEAR

    Actual Output: Solver.force == SHEAR

    Results Summary: Pass

    \item{Solver.UT5\\}
    
    Control: Automatic

    Initial State: Solver.force == SHEAR
    
    Method: static void updateForceType(std::string forceStr)
    
    Input: forceStr = "Positive Moment"
    
    Expected Output: Solver.force == POSITIVE\_MOMENT

    Actual Output: Solver.force == POSITIVE\_MOMENT

    Results Summary: Pass

    \item{Solver.UT6\\}
    
    Control: Automatic

    Initial State: Solver.force == POSITIVE\_MOMENT
    
    Method: static void updateForceType(std::string forceStr)
    
    Input: forceStr = "Negative"
    
    Expected Output: invalidConfigurationValue exception is thrown and Solver.force == POSITIVE\_MOMENT

    Actual Output: invalidConfigurationValue exception is thrown and Solver.force == POSITIVE\_MOMENT

    Results Summary: Pass

    \item{Solver.UT7\\}
    
    Control: Automatic

    Initial State: Solver.force == POSITIVE\_MOMENT
    
    Method: static void updateForceType(std::string forceStr)
    
    Input: forceStr = ""
    
    Expected Output: invalidConfigurationValue exception is thrown and Solver.force == POSITIVE\_MOMENT

    Actual Output: invalidConfigurationValue exception is thrown and Solver.force == POSITIVE\_MOMENT

    Results Summary: Pass

    \item{Solver.UT8\\}
    
    Control: Automatic

    Initial State: Solver.solver == CONCERNED
    
    Method: static void updateSolverType(std::string solverStr)
    
    Input: solverStr = "Critical Section"
    
    Expected Output: Solver.solver == CRITICAL

    Actual Output: Solver.solver == CRITICAL

    Results Summary: Pass

    \item{Solver.UT9\\}
    
    Control: Automatic

    Initial State: Solver.solver == CRITICAL
    
    Method: static void updateSolverType(std::string solverStr)
    
    Input: solverStr = "Concerned Section"
    
    Expected Output: Solver.solver == CONCERNED

    Actual Output: Solver.solver == CONCERNED

    Results Summary: Pass

    \item{Solver.UT10\\}
    
    Control: Automatic

    Initial State: Solver.solver == CONCERNED
    
    Method: static void updateSolverType(std::string solverStr)
    
    Input: solverStr = "Critical"
    
    Expected Output: invalidConfigurationValue exception is thrown and Solver.solver == CONCERNED

    Actual Output: invalidConfigurationValue exception is thrown and Solver.solver == CONCERNED

    Results Summary: Pass

    \item{Solver.UT11\\}
    
    Control: Automatic

    Initial State: Solver.solver == CONCERNED
    
    Method: static void updateSolverType(std::string solverStr)
    
    Input: solverStr = ""
    
    Expected Output: invalidConfigurationValue exception is thrown and Solver.solver == CONCERNED

    Actual Output: invalidConfigurationValue exception is thrown and Solver.solver == CONCERNED

    Results Summary: Pass

\end{enumerate}

    
\section{Changes Due to Testing}

The biweekly meetings with Dr Yang and her team have been wonderful for receiving feedback on the entire design process. Particularly once we had a working demo for Rev0.
As mentioned in the Design Verification section we had a focused meeting with Dr Yang and her team regarding the Rev 0 demo and what improvements they would like to see for
Rev1. One common thread that was exposed throughout the VnV process was our sore lack of \textbf{robustness}. This cost us dearly in the evaluation of Usability metrics that
was performed with Dr Yang and co. This is also the primary shortfall identified in the unit and system tests. The highest priority and most sweeping change 
that is planned to be made between Rev0 to the final product is definitely the addition of proper error checking and handling for many methods and processes in the program.\\

There were also some less catastrophic but still important findings as a result of the process. For example, Dr. Yang pointed out that our bridge visualization was inaccurate to real world
standards, due to the relative thinness of the middle bridge support and the lack of side supports and a bridge deck. This is valuable information that would not have been
easy to figure out without her feedback, and we will be changing this for Rev1.\\

Finally there were many small pieces of feedback regarding making the UI more intuitive/consistent/etc from Dr. Yang and her team. Things like wanting the concerned section
and discretization length information visible on the solver tab as well as the bridge tab, or change the labels on certain buttons or tabs to be more in line with industry
standards, to name a few.\\

\section{Automated Testing}
Automated testing was accomplished using the QTest framework and CTest. All of the unit testing performed for VnV was automated. The QTest framework was used to create the test suites as it was created for QT application and allows for easy execution of many tests while having a clear display of test results. CTest was used to augment the ease of use of QTest even further by integrating it with the CMake build system already in use. This combination allows for the straightforward and automated running of all unit tests.

\section{Trace to Requirements}

\begin{longtable}[H]{|p{0.2\textwidth} | p{0.8\textwidth}|}
  \caption{Functional System Test Traceability}
  \label{TblFRSTTraceability}\\
  \hline
  \textbf{Requirement} & \textbf{System Tests}\\
  \hline
  \endfirsthead
  \hline
  \textbf{Requirement} & \textbf{System Tests}\\
  \hline
  \endhead
  FR1 & FR1.ST1, FR1.ST2\\
  \hline
  FR2 & FR2.ST1, FR2.ST2\\
  \hline
  FR3 & FR3.ST1, FR3.ST2\\
  \hline
  FR4 & FR4.ST1\\
  \hline
  FR5 & FR5.ST1\\
  \hline
  FR6 & FR6.ST1, FR6.ST2\\
  \hline
  FR7 & FR7.ST1, FR7.ST2\\
  \hline
  FR8 & FR8.ST1\\
  \hline
  FR9 & FR9.ST1, FR9.ST2\\
  \hline
  FR10 & FR10.ST1\\
  \hline
  FR11 & FR11.ST1\\
  \hline
  FR12 & FR6.ST1\\
  \hline
  FR13 & FR13.ST1\\
  \hline
  FR14 & FR14.ST1\\
  \hline
  FR15 & FR15.ST1\\
  \hline
  FR16 & FR16.ST1\\
  \hline
\end{longtable}

\begin{longtable}[H]{|p{0.2\textwidth} | p{0.8\textwidth}|}
  \caption{Non-Functional System Test Traceability}
  \label{TblNFRSTTraceability}\\
  \hline
  \textbf{Requirement} & \textbf{System Tests}\\
  \hline
  \endfirsthead
  \hline
  \textbf{Requirement} & \textbf{System Tests}\\
  \hline
  \endhead
  NFR1 & NFR1.ST1\\
  \hline
  NFR2 & NFR2.ST1\\
  \hline
  NFR3 & NFR3.ST1, NFR3.ST2, NFR3.ST3, NFR3.ST4, NFR3.ST5\\
  \hline
  NFR4 & NFR4.ST1, NFR4.ST2, NFR4.ST3, NFR4.ST4\\
  \hline
  NFR5 & NFR5.ST1\\
  \hline
  NFR6 & NFR6.ST1\\
  \hline
  NFR7 & NFR7.ST1\\
  \hline
  NFR8 & NFR8.ST1\\
  \hline
  NFR9 & NFR9.ST1\\
  \hline
  NFR10 & NFR10.ST1\\
  \hline
  NFR11 & FR3.ST1, FR7.ST1, FR10.ST1, FR11.ST1, FR12.ST1, FR13.ST1\\
  \hline
  NFR12 & NFR9.ST1, NFR10.ST1\\
  \hline
  NFR13 & NFR13.ST1, NFR13.ST2, NFR13.ST3, NFR13.ST4\\
  \hline
  NFR14 & N/A\\
  \hline
  NFR15 & N/A\\
  \hline
\end{longtable}

\section{Trace to Modules}
The unit tests can be directly traced to modules by the name of the tests. For example, the test \hyperref[PlatoonConfiguration.UT7]{PlatoonConfiguration.UT7} is a unit test for the PlatoonConfiguration module.

\bibliographystyle{plainnat}
\bibliography{../../refs/References}

\newpage{}
\section*{Appendix --- Reflection}

The information in this section will be used to evaluate the team members on the
graduate attribute of Reflection.  Please answer the following question:

\begin{enumerate}
  \item In what ways was the Verification and Validation (VnV) Plan different
  from the activities that were actually conducted for VnV?  If there were
  differences, what changes required the modification in the plan?  Why did
  these changes occur?  Would you be able to anticipate these changes in future
  projects?  If there weren't any differences, how was your team able to clearly
  predict a feasible amount of effort and the right tasks needed to build the
  evidence that demonstrates the required quality?  (It is expected that most
  teams will have had to deviate from their original VnV Plan.)
\end{enumerate}

\textbf{Adham}: I think the biggest difference between what we planned for the VnV and what we actually managed to do was the extent of the involvement of outside parties. I don't think it was just patently infeasible, but due to time constraints we weren't really able to get user feedback from anyone except our supervisor Dr. Yang and her team. Granted, their feedback is by far the most important of anyone, but it would have been interesting and valuable to get feedback from other engineering students or even complete layman for example. Realistically this happened because we saved things for the last minute, and it could have been anticipated and avoided, but oh well.

\textbf{David}: The VnV Plan and actual VnV were most different in terms of how the testing was done. Our plan of interviews and surveys changed as we didn't have access to MTO bridge engineers, so we instead relied on feedback from Dr. Yang and her team, as well as a few friends (and family) who are in the civil engineering field. These changes occurred as we didn't have the time (in terms of deadlines) to accomplish this properly, so we resorted to less ideal options. It's fairly difficult to anticipate changes like this, sometimes life/work just get busy. To avoid more of the risk, I would come up with realistic contingency plans beforehand, so that we get the best feedback we can, with the resources we have.

\textbf{Pedram}: The team had truly laid out a clear process toward an encompassing verification plan however the scope and the room for error were a slightly misallocated. The original plan included surveys as well as interviews that were not quite realistic when combined with the actual implementation as well as other school work. In an ideal situation, there would be contingency plans in place that would allow quality work even if we were to miss internal milestones. Moving forward, there should be more frequent checkups within the team to make sure the tasks are being executed according to the plan. 

\textbf{Victor}: Another way in which the activities conducted for VnV were different to what was planned in the VnV Plan is the system tests. In the VnV Plan, many of the system tests were automated, but in actual testing many of these tests were conducted manually. This was because we had initially believed that the many GUI interactions required for the tests could be fairly easily automated through the use of QTEST, but when we got to actually testing the system we found that automating these interactions was infeasible. This change could be anticipated in the future by doing extensive research to fully understanding the capabilities and limitations of the testing tools and libraries being used before writing the VnV Plan.

\textbf{Farzad}: Another way in which the activities conducted for VnV were different to what was planned in the VnV Plan is implementation verification plan. Granted, that prototype testing ad unit testing was performed but there was a third dimensions to implementation verification which is analysis and it includes code inspection. The original plan was to have a system in which every newly written code would be reviewed and approved by other team members before it gets merged to the product. However, as a result of time pressure before preparing the first prototype team members were working in parallel and merging their continuously. More progress checking meetings could have potentially alerted us of this time crunch earlier and get us to start things off earlier and work slower with higher quality.

\textbf{Darren}: We stated we would survey civil engineers while they used the program to confirm things such as ease of use and intuitiveness but in practice settled for presenting the program to the client and her team and asking for feedback. This was done because it turned out difficult to get in touch with MTO and arrange for a meeting with a bridge engineer. We would've also needed access to their computers to fully follow the original plan. In the future we would settle for specifying the client's team as the ones surveyed instead or make earlier arrangements for meeting with bridge engineers.
\end{document}