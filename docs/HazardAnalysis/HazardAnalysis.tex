\documentclass{article}

\usepackage{booktabs}
\usepackage{tabularx}
\usepackage{hyperref}
\usepackage{float}
\usepackage{enumitem}
\hypersetup{
    colorlinks=true,       % false: boxed links; true: colored links
    linkcolor=red,          % color of internal links (change box color with linkbordercolor)
    citecolor=green,        % color of links to bibliography
    filecolor=magenta,      % color of file links
    urlcolor=cyan           % color of external links
}

\title{Hazard Analysis\\\progname}

\author{\authname}

\date{}

%% Comments

\usepackage{color}

\newif\ifcomments\commentstrue %displays comments
%\newif\ifcomments\commentsfalse %so that comments do not display

\ifcomments
\newcommand{\authornote}[3]{\textcolor{#1}{[#3 ---#2]}}
\newcommand{\todo}[1]{\textcolor{red}{[TODO: #1]}}
\else
\newcommand{\authornote}[3]{}
\newcommand{\todo}[1]{}
\fi

\newcommand{\wss}[1]{\authornote{blue}{SS}{#1}} 
\newcommand{\plt}[1]{\authornote{magenta}{TPLT}{#1}} %For explanation of the template
\newcommand{\an}[1]{\authornote{cyan}{Author}{#1}}

%% Common Parts

\newcommand{\progname}{MTOBridge} % PUT YOUR PROGRAM NAME HERE
\newcommand{\authname}{Team 15, Alpha Software Solutions
\\ Badawy, Adham
\\ Yazdinia, Pedram
\\ Jandric, David
\\ Vakili, Farzad
\\ Vezina, Victor
\\ Chiu, Darren} % AUTHOR NAMES                  

\usepackage{hyperref}
    \hypersetup{colorlinks=true, linkcolor=blue, citecolor=blue, filecolor=blue,
                urlcolor=blue, unicode=false}
    \urlstyle{same}


\begin{document}

\maketitle
\thispagestyle{empty}

~\newpage

\pagenumbering{roman}

\begin{table}[hp]
\caption{Revision History} \label{TblRevisionHistory}
\begin{tabularx}{\textwidth}{llX}
\toprule
\textbf{Date} & \textbf{Developer(s)} & \textbf{Change}\\
\midrule
October 12 2022 & Darren & Added System Boundaries \& Components\\
October 19 2022& Adham & Added Adham/Victor/Farzads FMEA work into a latex table\\
X & Y & Z\\
... & ... & ...\\
\bottomrule
\end{tabularx}
\end{table}

~\newpage

\tableofcontents

~\newpage

\pagenumbering{arabic}

\wss{You are free to modify this template.}

\section{Introduction}

This document is a hazard analysis of MTOBridge. A hazard is a potentially harmful event resulting
from the conditions of the system and the environment.

\section{Scope and Purpose of Hazard Analysis}

This document describes the components of MTOBridge, the potential hazards associated with each, and 
any new functional requirements that can be derived from these hazards. This process is important to 
identify any potential issues with the system, and then design the system to eliminate the potential issues.

\section{System Boundaries and Components}

This hazard analysis addresses the system that consists of the following components:

\begin{enumerate}
	\item UI Component, for providing a graphic display to the user and visualizing MATLAB results
	\item Input Handler Component, for processing user inputs
	\item MATLAB Interaction Component, for calling scripts and supplying specified arguments to them
	\item MATLAB Engine Component, for performing bridge calculations
	\item File Manager Component, for reading inputs from files and saving results in various formats
\end{enumerate}

The system boundary includes these software components and any dependency files required for the application to operate. Although the MATLAB Engine Component is owned by the client and its exact contents verified independent of this project, this hazard analysis will address it due to being a crucial component of the system.

\section{Critical Assumptions}

We will not be making any critical assumptions about the system. As mentioned above, the MATLAB engine 
is a critical component of the system. We must address it with the hazard analysis and will not make
any assumptions about its correctness and functionality.

\section{Failure Mode and Effect Analysis}
\begin{table}[H]
    \caption{FMEA Analysis} \label{TblFMEA}
    \begin{tabular}{|p{0.1\textwidth} | p{0.1\textwidth} | p{0.1\textwidth} | p{0.3\textwidth} | p{0.3\textwidth} | p{0.05\textwidth} | p{0.05\textwidth}|}
    \toprule
    \textbf{Comp onent} & \textbf{Failure} & \textbf{Effect} & \textbf{Cause} & \textbf{Recommended Action} & \textbf{SR} & \textbf{Ref}\\
    \midrule
    UI & UI Displays truck config incorrectly & User confusion / Misleading interface
&   \textbf{A.} Incorrect processing of regular user input---
    \textbf{B.} Runtime error in truck display module---
    \textbf{C.} Unexpected/boundary case user input 
&   \textbf{A.} Thoroughly test the truck display module to avoid unexpected responses to input---
    \textbf{B.}  Ensure truck display module has proper error handling to avoid catastrophic failure if an error is encountered, and perhaps instead prompt the user to try again, for example.---
    \textbf{C.} Design modules with separation of concerns in mind to limit complexity and increase program robustness.
&None&HA-1\\
    \midrule
    UI & UI does not update to match new truck config at all & User missing important information. & \textbf{A.} Failure to catch invalid user input.---
    \textbf{B.} Same as HA-1---
    \textbf{C.} Same as HA-1 
&   \textbf{A.} Ensure truck display module has proper input bounds and safety nets to catch invalid user inputs, instead of just running with them.---
    \textbf{B.} Same as HA-1---
    \textbf{C.} Same as HA-1
&None&HA-2\\
    \midrule
    UI & UI Displays bridge config incorrectly. & Same as HA-1 & \textbf{A.} Same as HA-1.---
    \textbf{B.} Same as HA-1.---
    \textbf{C.}  Same as HA-1.
&  \textbf{A.} Same as HA-1---
    \textbf{B.} Same as HA-1.---
    \textbf{C.} Same as HA-1.
    &None&HA-3\\
    \bottomrule
  \end{tabular}
  \end{table}
\pagebreak
\begin{table}[H]
    \begin{tabular}{|p{0.1\textwidth} | p{0.1\textwidth} | p{0.1\textwidth} | p{0.3\textwidth} | p{0.3\textwidth} | p{0.05\textwidth} | p{0.05\textwidth}|}
    \toprule
    \textbf{Comp onent} & \textbf{Failure} & \textbf{Effect} & \textbf{Cause} & \textbf{Recommended Action} & \textbf{SR} & \textbf{Ref}\\
\midrule
    UI & UI does not update to match new bridge config at all & Same as HA-2 &  \textbf{A.} Same as HA-2.---  
    \textbf{B.} Same as HA-2.---
    \textbf{C.} Same as HA-2.
&   \textbf{A.} Same as HA-2.---
    \textbf{B.}  Same as HA-2.---
    \textbf{C.} Same as HA-2.
    &None&HA-4\\
\midrule
    UI&UI attempts to display undesired calculation type&Display is worthless&\textbf{A.} Incorrect processing of regular user input.---
    \textbf{B.} Misleading or incorrect display of user solver selection.
    &\textbf{A.} Thoroughly test solver configuration module to incorrect processing of user input---
    \textbf{B.} Minimize complexity of input handler/solver selection display modules and the interaction between them to reduce chances of incorrect information passing
     and misleading or incorrect displays of input.
    &None&HA-5\\
\midrule
    UI&Truck platoon trip display does not match bridge load display.
    &Display is impossible to parse&\textbf{A.} Incorrect calculation display logic.---
    \textbf{B.} Unexpected bug or glitch is calculation display modules.
&\textbf{A.} Thoroughly test calculation display module(s) to avoid unexpected behavior.---
\textbf{B.} Include checks to determine if the two displays align and catch/correct it if they don’t instead of just displaying it anyways.---
\textbf{A/B.} Look into splitting the calculation display modules into two modules entirely, platoon trip and bridge load display, to simplify each part and 
reduce chance of logical errors.
&SR-1&HA-6\\
\bottomrule
  \end{tabular}
  \end{table}
  \pagebreak
  \begin{table}[H]
      \begin{tabular}{|p{0.1\textwidth} | p{0.1\textwidth} | p{0.1\textwidth} | p{0.3\textwidth} | p{0.3\textwidth} | p{0.05\textwidth} | p{0.05\textwidth}|}
      \toprule
      \textbf{Comp onent} & \textbf{Failure} & \textbf{Effect} & \textbf{Cause} & \textbf{Recommended Action} & \textbf{SR} & \textbf{Ref}\\
  \midrule
      UI & Platoon trip and Bridge load synch check(s) provides false positives or negatives &  Deny a fine display or let through an erroneous one.
    & \textbf{A.} Incorrect synch check logic.---
      \textbf{B.} Unexpected bug or glitch in synch module.
    & \textbf{A.} Thoroughly test synch checks to avoid unexpected behavior.---
      \textbf{B.} Simplify synch check logic as much as is possible while maintaining accuracy to limit chance of incorrect logic programming
    &SR-1&HA-7\\
  \midrule
      UI&UI incorrectly displays the concerned section&Display is worthless& \textbf{A.} unexpected bug/glitch in concerned section display module.---
      \textbf{B.} misleading or incorrect display of user concerned section selection.
    & \textbf{A.} Thoroughly test concerned section display module to incorrect processing of user input---
      \textbf{B.} Minimize complexity of input handler and concerned section display display modules and the interaction to reduce chances of 
      incorrect information passing and misleading/incorrect displays of input.
      &None&HA-8\\
  \midrule
      UI&UI incorrectly displays discretized bridge segments.
      &Same as HA-8&\textbf{A.} Same as HA-8.---
      \textbf{B.}Same as HA-8.
      &\textbf{A.} Same as HA-8.---
      \textbf{B.}Same as HA-8.
      &None&HA-9\\
  \bottomrule
    \end{tabular}
    \end{table}
    \pagebreak
    \begin{table}[H]
        \begin{tabular}{|p{0.1\textwidth} | p{0.1\textwidth} | p{0.1\textwidth} | p{0.3\textwidth} | p{0.3\textwidth} | p{0.05\textwidth} | p{0.05\textwidth}|}
        \toprule
        \textbf{Comp onent} & \textbf{Failure} & \textbf{Effect} & \textbf{Cause} & \textbf{Recommended Action} & \textbf{SR} & \textbf{Ref}\\
    \midrule
        UI &  UI fails to display calculation results entirely. &  Display is worthless.
      & \textbf{A.} Unexpected/boundary case user input.---               
        \textbf{B.} Runtime error in calculation display modules.---
        \textbf{C.} Failure to catch invalid user input

      & \textbf{A.} Thoroughly test the calculation display modules to avoid unexpected responses to input---
        \textbf{B.}  Ensure calculation display modules have proper error handling to avoid catastrophic failure if an error is encountered, and perhaps instead prompt
        the user to try again, for example.---
        \textbf{A/B.} Design modules with separation of concerns in mind to limit error propagation and increase program robustness.---
        \textbf{C.} Ensure calculation display modules have proper input bounds and safety nets to catch invalid user inputs, instead of just running with them.
      &None&HA-10\\
    \midrule
        UI&UI stops reacting to user inputs&User locked out from using UI, program is worthless
      & \textbf{A.} Runtime error in calculation display modules.---
        \textbf{B.} Parallel computing issue such as deadlock that hangs the program.
 
      & \textbf{A.}  Ensure all display modules have proper error handling to avoid catastrophic failure if an error is encountered, 
        and perhaps instead prompt the user to try again, for example.---
        \textbf{B.} Implement proper thread safety measures to avoid deadlocks and other such issues.---
        \textbf{A/B.} Design all modules with separation of concerns in mind to limit error propagation and increase program robustness.
      &None&HA-11\\
    \bottomrule
      \end{tabular}
      \end{table}
      \pagebreak
    \begin{table}[H]
        \begin{tabular}{|p{0.1\textwidth} | p{0.1\textwidth} | p{0.1\textwidth} | p{0.3\textwidth} | p{0.3\textwidth} | p{0.05\textwidth} | p{0.05\textwidth}|}
        \toprule
        \textbf{Comp onent} & \textbf{Failure} & \textbf{Effect} & \textbf{Cause} & \textbf{Recommended Action} & \textbf{SR} & \textbf{Ref}\\
    \midrule
    UI& UI encounters parallel computing issue such as deadlock/race condition.
    & Incorrect results or unexpected program behavior
    & \textbf{A.} Multiple threads modifying the same values/waiting on each other.---
    & \textbf{A.} Implement proper thready safety measures to avoid deadlocks and other such issues
    &None & HA-12\\
    \midrule
    Matlab Interaction& Data received is incorrectly formatted.
    & Program cannot function
    & \textbf{A.} One-time error in cross-program communication caused by outside factors (OS, hardware, etc.)---
      \textbf{B.} Bug or error in MATLAB engine
    & \textbf{A.} Try all calculations a second time when the first calculation fails---
      \textbf{B.} The MATLAB engine will be tested thoroughly to try to reduce the amount of bugs it has. 
      The program will always log in-depth error information and display an error message to the user telling them to contact the developers when 
      there is an issue with the MATLAB component.     
    &SR-3 &HB-1\\
    \midrule
    Matlab Interaction& Unable to call engine.
      & Same as HB-1
      & \textbf{A.} One-time error in cross-program communication caused by outside factors (OS, hardware, etc.)---
        \textbf{B.} Engine not installed / installed improperly 
      & \textbf{A.} Try all calculations a second time when the first calculation fails---
        \textbf{B.} The program will always display a message telling the user that they must install the MATLAB engine with a reference 
        to the installation section of the user manual when the MATLAB engine is not detected 
      &SR-3 & HB-2\\
    \bottomrule
      \end{tabular}
      \end{table}
      \pagebreak
      \begin{table}[H]
          \begin{tabular}{|p{0.1\textwidth} | p{0.1\textwidth} | p{0.1\textwidth} | p{0.3\textwidth} | p{0.3\textwidth} | p{0.05\textwidth} | p{0.05\textwidth}|}
          \toprule
          \textbf{Comp onent} & \textbf{Failure} & \textbf{Effect} & \textbf{Cause} & \textbf{Recommended Action} & \textbf{SR} & \textbf{Ref}\\
      \midrule
      Matlab Engine& The engine crashes unexpectedly.
      & Same as HB-1
      & \textbf{A.} One-time crash caused by outside factors (OS, hardware, etc.)---
        \textbf{B.} Bug or error in MATLAB engine      
      & Same as HB-1    
      &None&HC-1\\
      \midrule
      Matlab Engine& The engine calculations take more time than should be required (more than 1 second).
      & Program must wait for results
      & \textbf{A.} One-time error causing infinite looping caused by outside factors (OS, hardware, etc.)---
        \textbf{B.} Bug or error in MATLAB engine      
      & Same as HB-1    
      &None&HC-2\\
      \midrule
      Matlab Engine& Data received from the engine is incorrect (as in physically impossible).
      & Same as HB-1
      & \textbf{A.} One-time calculation error caused by outside factors (OS, hardware, etc.)---
        \textbf{B.} Bug or error in MATLAB engine.     
      & Same as HB-1    
      &None&HC-3\\
      \bottomrule
        \end{tabular}
        \end{table}
        \pagebreak
        \begin{table}[H]
            \begin{tabular}{|p{0.1\textwidth} | p{0.1\textwidth} | p{0.1\textwidth} | p{0.3\textwidth} | p{0.3\textwidth} | p{0.05\textwidth} | p{0.05\textwidth}|}
            \toprule
            \textbf{Comp onent} & \textbf{Failure} & \textbf{Effect} & \textbf{Cause} & \textbf{Recommended Action} & \textbf{SR} & \textbf{Ref}\\
        \midrule
        Input Handler& Handler passes inputs to other components that are too large or small.
        & Inaccurate analysis results
        & \textbf{A.} Accidental changes of input for example writing 100000 instead of 10000000      
        & \textbf{A.} Validating numeric values are within an acceptable range   
        &SR-4&HD-1\\
        \midrule
        Input Handler& Handler invariant to type changes.
        & System Crash
        & \textbf{A.} Accidental mix and match of inputs for example inputting structure material inside load section
        & \textbf{A.} Validating input type before passing it on.    
        &SR-5&HD-2\\
        \midrule
        Input Handler& Handler passes on incomplete set of inputs.
        & Same as HD-2
        & \textbf{A.} Submitting before completing all input sections     
        & \textbf{A.} Detecting if required inputs are missing from the model   
        &None&HD-3\\
        \midrule
        File Manager& File Manager loads corrupted configuration and saved files.
        & Inaccurate results or system crash
        & \textbf{A.} Process responsible for creating the file was interrupted.---
          \textbf{B.} files edited manually by power users
        & \textbf{A.} Have metrics that indicates file creation was completed and if not it is communicated to the user when loading.---
          \textbf{B.} Have metrics such as checksums to ensure the integrity of the files.  
        &SR-6&HE-1\\
        \midrule
        File Manager& File Manager partially saves file.
        & Data loss
        & \textbf{A.} Power outage---
          \textbf{B.} System crash        
        & \textbf{A/B.} Automatically save to a file whenever user changes the configuration or at reasonable time intervals  
        &SR-7&HE-2\\
        \bottomrule
          \end{tabular}
          \end{table}

      
 

\section{Safety and Security Requirements}
Using the results of FMEA, we can derive the following safety and security requirements for our system in order to mitigate the identified hazards. \\

\textbf{SR-1}:
The system must be able to synchronize the display of the truck platoon with the input bridge load through various logic checks and boundary tests. An alarm/error shall be produced if the display requirements across both modules are not met. 

\emph{Rationale}: While splitting the calculation display modules into two modules; platoon trip and bridge load display, can simplify each part, it can lead to mismatch in respective display.\\

\textbf{SR-2}: 
The system must produce a log of calls and functions with a detailed trace of function callbacks highlighting the code locations. The log will include timestamps along with software environment information such as input.

\emph{Rationale}:
The logs will be directly used in the debugging process which can be presented in different ways to the user. Such logs can prove useful in case of engine crashes, data loss and timeouts. \\

\textbf{SR-3}:
In case of an Matlab engine failure, the system must be able to use the recorded logs to provide a clear message to the user.

\emph{Rationale}: Logs while carrying a lot of information can be too heavy to digest, we can help the user further with a bit of processing.\\ 

\textbf{SR-4}:
The system must validate numeric values are within an acceptable range before being passed on to other components.

\emph{Rationale}: Accidental changes of input can cause massive shifts in the analysis and cause inaccurate results. (e.g. adding extra 0’s) \\

\textbf{SR-5}:
The system must validate input type before being passed on to other components.

\emph{Rationale}: This is to prevent accidental mix and match of inputs for example inputting structure material inside load section.\\

\textbf{SR-6}:
The system must include information that indicates the integrity of the file and whether the file creation is completed.

\emph{Rationale}: This can help remove the risks associated with corrupt files during an import or export. \\

\textbf{SR-7}:
The system must automatically save to a file whenever user changes the configuration or at reasonable time intervals

\emph{Rationale}: This is meant to mitigate risks associated with system crashes or power outages.\\


\wss{Newly discovered requirements.  These should also be added to the SRS.  (A
rationale design process how and why to fake it.)}

\section{Roadmap}

\wss{Which safety requirements will be implemented as part of the capstone timeline?
Which requirements will be implemented in the future?}

\end{document}