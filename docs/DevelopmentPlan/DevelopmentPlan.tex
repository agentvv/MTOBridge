\documentclass{article}

\usepackage{booktabs}
\usepackage{tabularx}

\title{Development Plan\\\progname}

\author{\authname}

\date{}

%% Comments

\usepackage{color}

\newif\ifcomments\commentstrue %displays comments
%\newif\ifcomments\commentsfalse %so that comments do not display

\ifcomments
\newcommand{\authornote}[3]{\textcolor{#1}{[#3 ---#2]}}
\newcommand{\todo}[1]{\textcolor{red}{[TODO: #1]}}
\else
\newcommand{\authornote}[3]{}
\newcommand{\todo}[1]{}
\fi

\newcommand{\wss}[1]{\authornote{blue}{SS}{#1}} 
\newcommand{\plt}[1]{\authornote{magenta}{TPLT}{#1}} %For explanation of the template
\newcommand{\an}[1]{\authornote{cyan}{Author}{#1}}

%% Common Parts

\newcommand{\progname}{MTOBridge} % PUT YOUR PROGRAM NAME HERE
\newcommand{\authname}{Team 15, Alpha Software Solutions
\\ Badawy, Adham
\\ Yazdinia, Pedram
\\ Jandric, David
\\ Vakili, Farzad
\\ Vezina, Victor
\\ Chiu, Darren} % AUTHOR NAMES                  

\usepackage{hyperref}
    \hypersetup{colorlinks=true, linkcolor=blue, citecolor=blue, filecolor=blue,
                urlcolor=blue, unicode=false}
    \urlstyle{same}


\begin{document}

\begin{table}[hp]
\caption{Revision History} \label{TblRevisionHistory}
\begin{tabularx}{\textwidth}{llX}
\toprule
\textbf{Date} & \textbf{Developer(s)} & \textbf{Change}\\
\midrule
September 21 2022 & Darren & Team information sections\\
September 21 2022 & David & Technology sections\\
Date2 & Name(s) & Description of changes\\
... & ... & ...\\
\bottomrule
\end{tabularx}
\end{table}

\newpage

\maketitle

\wss{Put your introductory blurb here.}

\section{Team Meeting Plan}

The team will meet on Tuesdays on campus at 5:30PM weekly for 1 to 1.5 hours. Team members will meet in-person when possible but will join virtually when unable to be physically present. Meetings are expected to review progress made since the previous meeting, review the status of the project, and discuss and divide upcoming work among team members.

Adham has reserved a room in Gerald Hatch Centre and will chair meetings. Darren will open the call for team members to join virtually. Meeting minutes will be taken as needed when group decisions are reached on topics, responsibility for which will pass between team members as is appropriate.

\section{Team Communication Plan}

Git issues will be used to track ongoing tasks on the project. Aside from meetings and impromptu conversations between shared classes, the team is communicating primarily through a group chat in Discord.

The team will communicate with the client via email and Zoom calls. Team members will CC the rest of the team when emailing the client.

\section{Team Member Roles}

\begin{itemize}
	\item Adham will act as team leader and chair meetings.
	\item Darren will facilitate virtual meetings and has experience in C++.
	\item David has experience in C++ and is most familiar with the UI implementation framework we currently expect to use.
	\item Farzad possesses the most domain knowledge in civil engineering and is the main point of contact with the client.
	\item Pedram will maintain the git repo.
	\item Victor has experience in C++.
\end{itemize}

\section{Workflow Plan}

\begin{itemize}
	\item How will you be using git, including branches, pull request, etc.?
	\item How will you be managing issues, including template issues, issue
	classificaiton, etc.?
\end{itemize}

The git will be separated into a dev and prod (master) branch. Developers will fork from the dev branch to implement changes. After merging changes into the dev branch and confirming no issues with integration, the dev branch will be merged into prod. CI/CD will be set up between prod and dev as well as between dev and feature branches.

\section{Proof of Concept Demonstration Plan}

What is the main risk, or risks, for the success of your project?  What will you
demonstrate during your proof of concept demonstration to convince yourself that
you will be able to overcome this risk?

\section{Technology}

\subsection{Programming Language}
C++20 will be used for the GUI, and will be connected to the backend made in MATLAB.

\subsection{Linter}
In order to maintain a consistent format, we will use clang-format for linting. Using this linter, 
we can also easily select a coding standard to follow, which will come from a file in our repo.

\begin{itemize}
\item Specific unit testing framework
\item Investigation of code coverage measuring tools
\item Specific plans for Continuous Integration (CI), or an explanation that CI
  is not being done
\item Specific performance measuring tools (like Valgrind), if
  appropriate
\end{itemize}

\subsection{Libraries}
A GUI framework called \href{https://www.qt.io/}{Qt} will be used.
\begin{itemize}
	\item This framework will allow for easy creation of a simple GUI, but also allows for extensibility which may be required.
	\item It contains basics for creating buttons, text fields etc., and more complex components like graphs (or even 3D views) with good performance.
	\item It also contains a testing framework for easily testing the GUI, as well as benchmarking performance.
\end{itemize}
MATLAB is being used for the backend, and in order to connect it with the GUI, libraries provided in MATLAB are needed in order to call MATLAB functions
from C++.

\subsection{Tools}
For building the project, CMake will be used. Also, CMake will be used for deploying the application.

\section{Coding Standard}

\section{Project Scheduling}

\wss{How will the project be scheduled?}

\end{document}
