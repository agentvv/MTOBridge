\documentclass[12pt, titlepage]{article}

\usepackage{booktabs}
\usepackage{tabularx}
\usepackage{hyperref}
\hypersetup{
    colorlinks,
    citecolor=black,
    filecolor=black,
    linkcolor=red,
    urlcolor=blue
}
\usepackage[round]{natbib}

%% Comments

\usepackage{color}

\newif\ifcomments\commentstrue %displays comments
%\newif\ifcomments\commentsfalse %so that comments do not display

\ifcomments
\newcommand{\authornote}[3]{\textcolor{#1}{[#3 ---#2]}}
\newcommand{\todo}[1]{\textcolor{red}{[TODO: #1]}}
\else
\newcommand{\authornote}[3]{}
\newcommand{\todo}[1]{}
\fi

\newcommand{\wss}[1]{\authornote{blue}{SS}{#1}} 
\newcommand{\plt}[1]{\authornote{magenta}{TPLT}{#1}} %For explanation of the template
\newcommand{\an}[1]{\authornote{cyan}{Author}{#1}}

%% Common Parts

\newcommand{\progname}{MTOBridge} % PUT YOUR PROGRAM NAME HERE
\newcommand{\authname}{Team 15, Alpha Software Solutions
\\ Badawy, Adham
\\ Yazdinia, Pedram
\\ Jandric, David
\\ Vakili, Farzad
\\ Vezina, Victor
\\ Chiu, Darren} % AUTHOR NAMES                  

\usepackage{hyperref}
    \hypersetup{colorlinks=true, linkcolor=blue, citecolor=blue, filecolor=blue,
                urlcolor=blue, unicode=false}
    \urlstyle{same}


\begin{document}

\title{Verification and Validation Report: \progname} 
\author{\authname}
\date{\today}
	
\maketitle

\pagenumbering{roman}

\section{Revision History}

\begin{tabularx}{\textwidth}{p{3cm}p{2cm}X}
\toprule {\bf Date} & {\bf Version} & {\bf Notes}\\
\midrule
Date 1 & 1.0 & Notes\\
Date 2 & 1.1 & Notes\\
\bottomrule
\end{tabularx}

~\newpage

\section{Symbols, Abbreviations and Acronyms}

\renewcommand{\arraystretch}{1.2}
\begin{tabular}{l l} 
  \toprule		
  \textbf{symbol} & \textbf{description}\\
  \midrule 
  T & Test\\
  \bottomrule
\end{tabular}\\

\wss{symbols, abbreviations or acronyms -- you can reference the SRS tables if needed}

\newpage

\tableofcontents

\listoftables %if appropriate

\listoffigures %if appropriate

\newpage

\pagenumbering{arabic}

This document serves as a plan to verify and validate the design documents and outlines a series of system and unit tests for the functional and non functional requirements
of the program. It is intended to be implemented as written, and for its results to be reported on in a future document.

\section{General Information}

\subsection{Summary}

The software whose Verification and Validation is planned for and referred to throughout this document is the User Interface component of MTO Bridge, 
a software developed in partnership with the Civil Engineering Department at McMaster university to assist MTO engineers in load rating bridges.
Its primary function is to present visually, in an intuitive and digestible way, the results of a simulation written by a Civil Engineering Professor and her lab 
in MATLAB of the forces imparted on a bridge by a platoon of trucks driving over it. There are two different solvers for two different results about the forces 
imparted on the bridge. The program will allow users to specify details of the bridge, the truck platoon driving over it, and which of the two results they are 
interested in knowing more about.\\

\subsection{Objectives}

There are two overarching objectives to any V\&V plan, ensuring we are actually building what we want to build correctly, or verification,
and ensuring that we think we want to build is what the client actually wants from us, or validation. Verification is where a lot of the domain specific definitions
of "correct" actually come in. As far as the verification plans for the design documents, the main goal is to build confidence that the design documents are complete,
verifiable, unambiguous, and generally of high quality. The design documents inform the implementation to a huge degree, so ensuring we are building on a solid foundation
is one of the main goals of this plan. As we are building a user interface, non functional qualities such as usability and performance/responsiveness are of 
utmost importance, and many of the tests outlined in this document will center around building confidence that our program has those qualities, or at least 
exposing if it doesn't early on. As well, like any program, a number of our tests focus around ensuring that our program as written actually does what we think it does,
in a predictable way. Ensuring our program properly executes all the functions we laid out in the design documents, and nothing more or less.\\


\subsection{Relevant Documentation}

Essentially every design document written for this project will be relevant documentation here, a mostly exhaustive list of this would include:
\href{https://github.com/agentvv/MTOBridge/blob/main/docs/ProblemStatementAndGoals/ProblemStatement.pdf}{The Problem Statement}\\
\href{https://github.com/agentvv/MTOBridge/blob/main/docs/DevelopmentPlan/DevelopmentPlan.pdf}{The Development Plan}\\
\href{https://github.com/agentvv/MTOBridge/blob/main/docs/SRS/SRS.pdf}{The SRS}\\
\href{https://github.com/agentvv/MTOBridge/blob/main/docs/HazardAnalysis/HazardAnalysis.pdf}{The Hazard Analysis}\\
The Module Guide: This document is not yet written, but will be extremely relevant for this plan, particularly for unit testing\\
The Module Interface Spec: Same as the Module Guide.\\

\section{Document Evaluation}

\subsection{SRS Evaluation}

\subsection{Design Evaluation}

\subsection{Verification and Validation Plan Evaluation}

\subsubsection{Completeness}

The primary portion of the V\&V Plan that is not complete is the section on unit testing. While many unit tests have now been written and performed, the V\&V Plan was written at a time when no unit tests were created. The now completed unit tests will be added into the V\&V Plan document so that this section can be fully complete once the project is over. Besides this issue, the V\&V Plan is also missing system tests designed to test NFR.14 and NFR.15. These tests are either going to be added or the NFRs will be reevaluated. Section 2 of the V\&V Plan is also missing a number of abbreviations that are used later in the document.

\section{System Evaluation}

\subsection{Functional Requirements Evaluation}

\subsubsection{MATLAB}

\paragraph{MATLAB Communication}

\begin{enumerate}

  \item{FR1.ST1\\}

  Control: Automatic

  Initial State: None

  Input: None

  Output: MATLAB engine started

  Test Case Derivation: The first step in the communication is to get the MATLAB engine
  to start.

  How test will be performed: The test will instantiate our MATLAB engine wrapper.

  \item{FR1.ST2\\}

  Control: Automatic

  Initial State: None

  Input: Truck, Bridge, and Analysis configurations

  Output: Any result from standard out or standard error

  Test Case Derivation: We just need to confirm that the engine is processing our
  input in some way, whether it gives an error or not.

  How test will be performed: Start the MATLAB engine through our wrapper, create
  some default truck, bridge, and analysis configurations, and try to run the analysis.

\end{enumerate}

\subsubsection{Truck Configuration}

\paragraph{Truck Configuration Input}

\begin{enumerate}

  \item{FR2.ST1\\}

  Control: Automatic

  Initial State: GUI exists

  Input: Axle load, axle spacing, number of trucks, headway

  Output: GUI fields are filled

  Test Case Derivation: If the data of the input fields can be entered, then the user
  should also be able to fill in the fields.

  How test will be performed: Input fields will filled, and checked if they have the
  correct value.

  \item{FR2.ST2\\}

  Control: Automatic

  Initial State: GUI exists

  Input: Invalid axle load, axle spacing, number of trucks, headway

  Output: GUI fields are filled, highlighted as incorrect

  Test Case Derivation: The user should be able to enter data, but be informed if the
  entered information is invalid.

  How test will be performed: Input fields will filled with incorrect information, 
  and the fields should be highlighted as incorrect.

\end{enumerate}

\paragraph{Truck Platoon Visualization}

\begin{enumerate}

  \item{FR3.ST1\\}

  Control: Automatic

  Initial State: GUI exists

  Input: Valid truck configuration

  Output: A valid visualization of the truck platoon shown

  Test Case Derivation: If a valid truck configuration is given, the user should be able to
  visualize the truck platoon.

  How test will be performed: Given a valid truck configuration, a visualization will be generated.
  The visualization will be saved as a picture and compared to the expected visualization.

  \item{FR3.ST2\\}

  Control: Automatic

  Initial State: GUI exists

  Input: Invalid truck configuration

  Output: A message informing the user of an incorrect truck configuration

  Test Case Derivation: If an invalid truck configuration is given, the user should be informed
  that the visualization cannot be created without a valid truck configuration.

  How test will be performed: Given an invalid configuration, the program will not show
  a visualization, but a message informing the user that a visualization cannot be created/shown
  until the configuration is valid.

\end{enumerate}

\paragraph{Save Truck Configuration}

\begin{enumerate}

  \item{FR4.ST1\\}

  Control: Automatic

  Initial State: GUI exists

  Input: A truck configuration

  Output: A file with the correctly saved configuration

  Test Case Derivation: The program should be able to take the existing truck configuration and
  save it so that it can be loaded or used later.

  How test will be performed: Given a truck configuration, a file will be created. Its contents
  will be compared with the expected contents.

\end{enumerate}

\paragraph{Load Truck Configuration}

\begin{enumerate}

  \item{FR5.ST1\\}

  Control: Automatic

  Initial State: GUI exists, a truck configuration file exists

  Input: Truck configuration file

  Output: Truck configuration input fields will be filled correctly

  Test Case Derivation: After loading a previously saved truck configuration file, the configuration
  should apply and fill the input fields for the truck configuration.

  How test will be performed: Given a truck configuration, a file will be created. Its contents
  will be compared with the expected contents.

\end{enumerate}

\subsubsection{Bridge Configuration}

\paragraph{Bridge Configuration Input}

\begin{enumerate}

  \item{FR6.ST1\\}

  Control: Automatic
  
  Initial State: GUI exists
  
  Input: Number of spans, bridge length
  
  Output: GUI fields are filled
  
  Test Case Derivation: If the data of the input fields can be entered, then the user
  should also be able to fill in the fields.
  
  How test will be performed: Input fields will filled, and checked if they have the
  correct value.
  
  \item{FR6.ST2\\}
  
  Control: Automatic
  
  Initial State: GUI exists
  
  Input: Invalid number of spans, bridge length
  
  Output: GUI fields are filled, highlighted as incorrect
  
  Test Case Derivation: The user should be able to enter data, but be informed if the
  entered information is invalid.
  
  How test will be performed: Input fields will filled with incorrect information, 
  and the fields should be highlighted as incorrect.

\end{enumerate}

\paragraph{Bridge Visualization}

\begin{enumerate}
  
  \item{FR7.ST1\\}
  
  Control: Automatic
  
  Initial State: GUI exists
  
  Input: Valid bridge configuration
  
  Output: A valid visualization of the bridge shown
  
  Test Case Derivation: If a valid bridge configuration is given, the user should be able to
  visualize the bridge.
  
  How test will be performed: Given a valid bridge configuration, a visualization will be generated.
  The visualization will be saved as a picture and compared to the expected visualization.
  
  \item{FR7.ST2\\}
  
  Control: Automatic
  
  Initial State: GUI exists
  
  Input: Invalid bridge configuration
  
  Output: A message informing the user of an incorrect bridge configuration
  
  Test Case Derivation: If an invalid bridge configuration is given, the user should be informed
  that the visualization cannot be created without a valid bridge configuration.
  
  How test will be performed: Given an invalid configuration, the program will not show
  a visualization, but a message informing the user that a visualization cannot be created/shown
  until the configuration is valid.
  
\end{enumerate}

\paragraph{Save Bridge Configuration}

\begin{enumerate}

  \item{FR8.ST1\\}
  
  Control: Automatic
  
  Initial State: GUI exists
  
  Input: A bridge configuration
  
  Output: A file with the correctly saved configuration
  
  Test Case Derivation: The program should be able to take the existing bridge configuration and
  save it so that it can be loaded or used later.
  
  How test will be performed: Given a bridge configuration, a file will be created. Its contents
  will be compared with the expected contents.

\end{enumerate}

\paragraph{Load Bridge Configuration}

\begin{enumerate}
  
  \item{FR9.ST1\\}
  
  Control: Automatic
  
  Initial State: GUI exists, a bridge configuration file exists
  
  Input: Bridge configuration file
  
  Output: Bridge configuration input fields will be filled correctly
  
  Test Case Derivation: After loading a previously saved truck configuration file, the configuration
  should apply and fill the input fields for the truck configuration.
  
  How test will be performed: Given a bridge configuration, a file will be created. Its contents
  will be compared with the expected contents.

\end{enumerate}

\subsubsection{Solver Setup}

\paragraph{Solver Selection}

\begin{enumerate}

  \item{FR10.ST1\\}

  Control: Automatic
            
  Initial State: None
            
  Input: input to indicate the desired solver 
            
  Output: Calculation results based on the selected solver

  Test Case Derivation: The current logic allows for calculation of the demand at a concerned section, or at the critical section, and there are separate calculations for each.

  How test will be performed: The test involves running a series of pre-determined calculations using both solvers and then comparing with the expected output. 
					
\end{enumerate}

\paragraph{Section of Concern}

\begin{enumerate}

  \item{FR11.ST1\\}

  Control: Automatic
            
  Initial State: GUI exists
            
  Input: location of concerned section
            
  Output: GUI field is filled

  Test Case Derivation: If the data of the input fields can be entered, then the user should also be able to fill in the fields.

  How test will be performed: Input fields will be filled, and checked if they have the correct value. 
					
\end{enumerate}

\paragraph{Discretization Length}

\begin{enumerate}

  \item{FR12.ST1\\}

Control: Automatic
					
Initial State: GUI exists
					
Input: discretization length 
					
Output: GUI field is filled

Test Case Derivation: If the data of the input fields can be entered, then the user should also be able to fill in the fields.

How test will be performed: Input fields will be filled, and checked if they have the correct value. 

\end{enumerate}

\paragraph{Force Type}

\begin{enumerate}

\item{FR13.ST1\\}

Control: Automatic
					
Initial State: GUI exists
					
Input: force type  
					
Output: GUI field is filled

Test Case Derivation: The user can select the type of force to calculate for. If they select the force type it should reflect in the GUI by filling the field properly.

How test will be performed: Input fields will filled, and checked if they have the correct value.
					
\item{FR13.ST2\\}

Control: Automatic
					
Initial State: None
					
Input: text input to indicate the desired moment  
					 
Output: Calculation results based on the selected parameters 

Test Case Derivation: The current implementation of the engine allows for calculation using positive and negative moment 

How test will be performed: The test will be performed by using our pool of pre-determined bridge and truck platoon configs to test for both positive and negative moment given a certain force type. 

\end{enumerate}

\subsubsection{Result Visualization}

\paragraph{Concerned Section Result Visualization}

\begin{enumerate}

\item{FR14.ST1\\}

Control: Automatic
				
Initial State: None
					
Input: calculations based on the concerned section 
					
Output: Animation and visualization highlighting the resulting properties on the specified bridge and load 

Test Case Derivation: As one of the main purposes of the program, the visualization will be interactive to show various views

How test will be performed: Relying on the pre-determined and validated data, we will graph the visualizations of each through the system as well as manually through the data produced by the engine. We can then compare the the expected and actual result across all available views. 
			
\end{enumerate}

\paragraph{Critical Section Result Visualization}

\begin{enumerate}

\item{FR15.ST1\\}

Control: Automatic
					
Initial State: None
					
Input: calculations based on the discretization length 
					
Output:  Animation and visualization highlighting the resulting properties on the specified bridge and load 

Test Case Derivation: As one of the main purposes of the program, the visualization will be interactive to show various views

How test will be performed: Relying on the pre-determined and validated data, we will graph the visualizations of each through the system as well as manually through the data produced by the engine. We can then compare the the expected and actual result across all available views. 

\end{enumerate}

\subsubsection{Exectution Flow Logging}

\paragraph{Report Generation}

\begin{enumerate}

\item{FR16.ST1\\}

Control: Automatic
					
Initial State: None
					
Input: text input to indicate logging flags 
					
Output: text output, "logs"

Test Case Derivation: With the engine built in another framework, its important to log every state change throughout the calculations as triggered by the system

How test will be performed: While running multiple calculations, the system is asked to comprehensively log the flow in the engine. The system will be interrupted at times to test the robustness as well. The resulting logs should be clearly highlighting the steps taken including any errors or warnings.
					
\end{enumerate}

\subsection{Nonfunctional Requirements Evaluation}

\subsubsection{Look and Feel Requirements}

\begin{enumerate}

\item{NFR1.ST1\\}

Type: Manual, Static

Initial State: Bridge UI mockups containing graphic elements will be prepared

Input/Condition: Civil engineers are presented with a UI mockup as seen in Figure \ref{fig:ui-mockup} 

Output/Results: The engineers can correctly identify what part of a bridge each UI element corresponds to in 90\% of cases.

How test will be performed: The developers will present a UI mockup to civil engineers and will assess how accurately the engineers can identify the UI elements.

\end{enumerate}

\subsubsection{Usability and Humanity Requirements}
		
\begin{enumerate}

\item{NFR2.ST1\\}

Type: Manual, Dynamic

Initial State: Application is opened with no past data

Input/Condition: Civil engineers are presented with the application and a brief explanation

Output/Results: 90\% of the civil engineers are able to perform bridge analysis within 5 minutes of introduction

How test will be performed: Civil engineers will be presented with the application and a brief explanation by the developers, then the amount of time it takes them to complete a bridge analysis will be measured.

\item{NFR3.ST1\\}

Type: Manual, Dynamic

Initial State: Application is opened

Input/Condition: Application is running on a computer with 1280x720 display

Output/Results: All buttons and animation window remain visible and accessible

How test will be performed: The application will be manually opened on a 1280x720 display and checked to ensure that all buttons and animation windows are still visible.

\item{NFR3.ST2\\}

Repeat test NFR.ST3 for 1366x768 display

\item{NFR3.ST3\\}

Repeat test NFR.ST3 for 1920x1080 display

\item{NFR3.ST4\\}

Repeat test NFR.ST3 for 2560x1440 display

\item{NFR3.ST5\\}

Repeat test NFR.ST3 for 3840x2160 display

\item{NFR4.ST1\\}

Type: Manual, Dynamic

Initial State: Application is opened with no past data

Input/Condition: Font size setting is changed to 8pt

Output/Results: All buttons and animation window remain visible and accessible

How test will be performed: The application will be manually opened and set to use 8pt font, then checked to ensure that all buttons and animation windows are still visible.

\item{NFR4.ST2\\}

Repeat test NFR.ST8 with font size 16pt

\item{NFR4.ST3\\}

Repeat test NFR.ST8 with font size 24pt

\item{NFR4.ST4\\}

Repeat test NFR.ST8 with font size 32pt

\item{NFR5.ST1\\}

Type: Manual, Dynamic

Initial State: Application is not installed

Input/Condition: Application is downloaded and installed

Output/Results: Application is ready to use within 30 minutes

How test will be performed: The application will be manually downloaded (on an internet connection with at least a 10Mbps download speed), installed, and run. This process will be timed to ensure that it takes less than 30 minutes.

\item{NFR6.ST1\\}

Type: Manual, Static

Initial State: Visually similar UI elements are grouped into categories by developers

Input/Condition: Civil engineer are asked to associate UI elements with their respective categories

Output/Results: Civil engineer sorts at least 90\% of UI elements into their predefined categories

How test will be performed: The developers will present a UI mockup to civil engineers and will assess how accurately the engineers can correctly group the UI elements.

\end{enumerate}

\subsubsection{Performance Requirements}

\begin{enumerate}

\item{NFR7.ST1\\}

Type: Automatic, Dynamic

Initial State: Application is opened with no past data

Input/Condition: The application is run with invalid inputs (zeroes, negative numbers, strings, etc.)

Output/Results: The application does not freeze or crash

How test will be performed: The application will be automatically run with many different invalid inputs, and it will be ensured that the application does not freeze or crash

\item{NFR8.ST1\\}

Type: Manual, Dynamic

Initial State: Application is installed but the MATLAB files are not installed

Input/Condition: The application is opened

Output/Results: The application will display an error regarding the missing MATLAB files

How test will be performed: The application will be manually installed without the required MATLAB files, and then it will be opened

\item{NFR9.ST1\\}

Type: Automatic, Dynamic

Initial State: Application is opened with no past data

Input/Condition: Various UI elements are interacted with

Output/Results: The UI elements react to the interaction within 100ms

How test will be performed: Various interactions with UI elements will be simulated on a computer with hardware similar to that of an MTO engineer's, and the speed of the UI response will be measured automatically

\item{NFR10.ST1\\}

Type: Automatic, Dynamic

Initial State: Application is opened with no past data

Input/Condition: Various valid inputs

Output/Results: Total execution time of calculations will not exceed underlying MATLAB script's execution time by more than 10\%

How test will be performed: Many different analyses will be performed and timed on a computer with hardware similar to that of an MTO engineer's.

\end{enumerate}

\subsubsection{Maintainability and Support Requirements}

\begin{enumerate}

\item{NFR13.ST1\\}

Type: Automatic, Static

Initial State: The program code is contained in multiple files

Input/Condition: Measure code file length in lines

Output/Results: At least 75\% of files will contain 750 lines of code or less

How test will be performed: The clang-tidy linter will be used to automatically measure file length

\item{NFR13.ST2\\}

Type: Automatic, Static

Initial State: The program code is contained in multiple files

Input/Condition: Measure method length in lines

Output/Results: At least 75\% of methods will contain 75 lines of code or less

How test will be performed: The clang-tidy linter will be used to automatically measure method length

\item{NFR13.ST3\\}

Type: Automatic, Static

Initial State: The program code is contained in multiple files

Input/Condition: Measure length of code lines in characters

Output/Results: At least 75\% of lines will contain 120 characters or less

How test will be performed: The clang-tidy linter will be used to automatically measure line length

\item{NFR13.ST4\\}

Type: Automatic, Static

Initial State: The program code is contained in multiple files

Input/Condition: Measure nesting depth of methods

Output/Results: At least 75\% of methods will have a maximum nesting depth of 5 or less

How test will be performed: The clang-tidy linter will be used to automatically measure method nesting depth

\end{enumerate}

\section{Unit Testing}
	
\section{Comparison to Existing Implementation}	

This section will not be appropriate for every project.

\section{Changes Due to Testing}

\wss{This section should highlight how feedback from the users and from 
the supervisor (when one exists) shaped the final product.  In particular 
the feedback from the Rev 0 demo to the supervisor (or to potential users) 
should be highlighted.}

\section{Automated Testing}
		
\section{Trace to Requirements}
		
\section{Trace to Modules}		

\section{Code Coverage Metrics}

\bibliographystyle{plainnat}
\bibliography{../../refs/References}

\newpage{}
\section*{Appendix --- Reflection}

The information in this section will be used to evaluate the team members on the
graduate attribute of Reflection.  Please answer the following question:

\begin{enumerate}
  \item In what ways was the Verification and Validation (VnV) Plan different
  from the activities that were actually conducted for VnV?  If there were
  differences, what changes required the modification in the plan?  Why did
  these changes occur?  Would you be able to anticipate these changes in future
  projects?  If there weren't any differences, how was your team able to clearly
  predict a feasible amount of effort and the right tasks needed to build the
  evidence that demonstrates the required quality?  (It is expected that most
  teams will have had to deviate from their original VnV Plan.)
\end{enumerate}

\end{document}