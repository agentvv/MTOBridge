\documentclass[12pt, titlepage]{article}

\usepackage{amsmath, mathtools}

\usepackage[round]{natbib}
\usepackage{amsfonts}
\usepackage{amssymb}
\usepackage{graphicx}
\usepackage{colortbl}
\usepackage{xr}
\usepackage{hyperref}
\usepackage{longtable}
\usepackage{xfrac}
\usepackage{tabularx}
\usepackage{float}
\usepackage{siunitx}
\usepackage{booktabs}
\usepackage{multirow}
\usepackage[section]{placeins}
\usepackage{caption}
\usepackage{fullpage}

\hypersetup{
bookmarks=true,     % show bookmarks bar?
colorlinks=true,       % false: boxed links; true: colored links
linkcolor=red,          % color of internal links (change box color with linkbordercolor)
citecolor=blue,      % color of links to bibliography
filecolor=magenta,  % color of file links
urlcolor=cyan          % color of external links
}

\usepackage{array}

\externaldocument{../../SRS/SRS}

%% Comments

\usepackage{color}

\newif\ifcomments\commentstrue %displays comments
%\newif\ifcomments\commentsfalse %so that comments do not display

\ifcomments
\newcommand{\authornote}[3]{\textcolor{#1}{[#3 ---#2]}}
\newcommand{\todo}[1]{\textcolor{red}{[TODO: #1]}}
\else
\newcommand{\authornote}[3]{}
\newcommand{\todo}[1]{}
\fi

\newcommand{\wss}[1]{\authornote{blue}{SS}{#1}} 
\newcommand{\plt}[1]{\authornote{magenta}{TPLT}{#1}} %For explanation of the template
\newcommand{\an}[1]{\authornote{cyan}{Author}{#1}}

%% Common Parts

\newcommand{\progname}{MTOBridge} % PUT YOUR PROGRAM NAME HERE
\newcommand{\authname}{Team 15, Alpha Software Solutions
\\ Badawy, Adham
\\ Yazdinia, Pedram
\\ Jandric, David
\\ Vakili, Farzad
\\ Vezina, Victor
\\ Chiu, Darren} % AUTHOR NAMES                  

\usepackage{hyperref}
    \hypersetup{colorlinks=true, linkcolor=blue, citecolor=blue, filecolor=blue,
                urlcolor=blue, unicode=false}
    \urlstyle{same}


\begin{document}

\title{Module Interface Specification for \progname{}}

\author{\authname}

\date{\today}

\maketitle

\pagenumbering{roman}

\section{Revision History}

\begin{tabularx}{\textwidth}{p{3cm}p{7cm}X}
\toprule {\bf Date} & {\bf  Developer(s)} & {\bf Notes}\\
\midrule
January 15, 2023 & David and Darren & Initial Draft\\
January 17, 2023 & Adham, Farzad, Pedram, and Victor & Conversion to LaTeX\\
January 18, 2023 & David and Victor & Formatting\\
April 4, 2023 & Darren & Rev1 Changes\\
\bottomrule
\end{tabularx}

~\newpage

\section{Symbols, Abbreviations and Acronyms}

See SRS Documentation at \url{https://github.com/agentvv/MTOBridge/blob/main/docs/SRS/SRS.pdf}.

\subsection{Abbreviations and Acronyms}
\renewcommand{\arraystretch}{1.2}
\begin{tabular}{l l} 
  \toprule		
  \textbf{symbol} & \textbf{description}\\
  \midrule
  MIS & Module Interface Specification \\
  SRS & Software Requirements Specification\\
  \bottomrule
\end{tabular}\\

\newpage

\tableofcontents

\newpage

\pagenumbering{arabic}

\section{Introduction}

The following document details the Module Interface Specifications for
\progname{}, 

Complementary documents include the System Requirement Specifications
and Module Guide.  The full documentation and implementation can be
found at \url{https://github.com/agentvv/MTOBridge}.

\section{Notation}

The structure of the MIS for modules comes from \citet{HoffmanAndStrooper1995},
with the addition that template modules have been adapted from
\cite{GhezziEtAl2003}.  The mathematical notation comes from Chapter 3 of
\citet{HoffmanAndStrooper1995}.  For instance, the symbol := is used for a
multiple assignment statement and conditional rules follow the form $(c_1
\Rightarrow r_1 | c_2 \Rightarrow r_2 | ... | c_n \Rightarrow r_n )$.

The following table summarizes the primitive data types used by \progname. 

\begin{center}
\renewcommand{\arraystretch}{1.2}
\noindent 
\begin{tabular}{l l p{7.5cm}} 
\toprule 
\textbf{Data Type} & \textbf{Notation} & \textbf{Description}\\ 
\midrule
character & char & a single symbol or digit\\
integer & $\mathbb{Z}$ & a number without a fractional component in (-$\infty$, $\infty$) \\
natural number & $\mathbb{N}$ & a number without a fractional component in [1, $\infty$) \\
real number & $\mathbb{R}$ & any number in (-$\infty$, $\infty$)\\
\bottomrule
\end{tabular} 
\end{center}

\noindent
The specification of \progname \ uses some derived data types: sequences, strings, and
tuples. Sequences are lists filled with elements of the same data type. strings
are sequences of characters. Tuples contain a list of values, potentially of
different types. In addition, \progname \ uses functions, which
are defined by the data types of their inputs and outputs. Local functions are
described by giving their type signature followed by their specification.

\section{Module Decomposition}

The following is taken directly from the Module Guide document for this project.

\begin{enumerate}
    \item[|] \textbf{Hardware Hiding}
    \begin{enumerate}
        \item[|] Data Storage Hiding
        \begin{enumerate}
            \item[|] Input Data Storage Hiding
            \begin{enumerate}
                \item[|] \textbf{Truck Platoon Configuration Data Storage Hiding}
                \item[|] \textbf{Bridge Configuration Data Storage Hiding}
            \end{enumerate}
            \item[|] Calculation Data Storage Hiding
            \begin{enumerate}
                \item[|] Calculation Variable Data Storage Hiding
                \begin{itemize}
                    \item[|] \textbf{Solver Choice Data Storage Hiding}
                    \item[|] \textbf{Concerned Section Specification Data Storage Hiding}
                    \item[|] \textbf{Critical Section Specification Data Storage Hiding}
                \end{itemize}
                \item[|] \textbf{Output Report Data Storage Hiding}
            \end{enumerate}
        \end{enumerate}
        \item[|] File System Hiding
        \begin{enumerate}
            \item[|] \textbf{Save To File System Hiding}
            \item[|] \textbf{Load From File System Hiding} 
        \end{enumerate}
    \end{enumerate}
    \item[|] \textbf{Software Decision Hiding}
    \begin{enumerate}
        \item[|] \textbf{Calculation Call Hiding}
        \item[|] \textbf{MATLAB Code Hiding}
        \item[|] Data Format Hiding
        \begin{enumerate}
            \item[|] Configuration Data Format Hiding
            \begin{itemize}
                \item[|] \textbf{Truck Platoon Configuration Data Format Hiding}
                \item[|] \textbf{Bridge Configuration Data Format Hiding}
                \item[|] \textbf{Solver Configuration Data Format Hiding}
            \end{itemize}
            \item[|] \textbf{Report Data Format Hiding}
        \end{enumerate}
    \end{enumerate}
    \item[|] \textbf{Behavior Hiding}
    \begin{enumerate}
        \item[|] Visualization Hiding
        \begin{enumerate}
            \item[|] Configuration Visualization Hiding
            \begin{itemize}
                \item[|] \textbf{Truck Platoon Configuration Visualization Hiding}
                \item[|] \textbf{Bridge Configuration Visualization Hiding}
            \end{itemize}
            \item[|] Calculation Visualization Hiding
            \begin{itemize}
                \item[|] \textbf{Calculation Call Visualization Hiding}
                \item[|] \textbf{Solver Configuration Visualization Hiding}
            \end{itemize}
            \item[|] \textbf{Output Report Visualization Hiding}
            \item[|] File System Visualization Hiding
            \begin{itemize}
                \item[|] {Save To File System Visualization Hiding}
                \begin{itemize}
                    \item[|] \textbf{Truck Platoon Configuration Save To File System Visualization Hiding}
                    \item[|] \textbf{Bridge Configuration Save To File System Visualization Hiding}
                    \item[|] \textbf{Solver Configuration Save To File System Visualization Hiding}
                    \item[|] \textbf{Output Report Save To File System Visualization Hiding}
                \end{itemize}
                \item[|] {Load From File System Visualization Hiding}
                \begin{itemize}
                    \item[|] \textbf{Truck Platoon Configuration Load From File System Visualization Hiding}
                    \item[|] \textbf{Bridge Configuration Load From File System Visualization Hiding}
                    \item[|] \textbf{Output Report Load From File System Visualization Hiding}
                \end{itemize}
            \end{itemize}
        \end{enumerate}
    \end{enumerate}
\end{enumerate}

~\newpage

\section{MIS of Save to File System}\label{Saver}

\subsection{Module}

Saver

\subsection{Uses}
\hyperref[PlatoonConfiguration]{PlatoonConfiguration}, \hyperref[BridgeConfiguration]{BridgeConfiguration}, \hyperref[SolverConfiguration]{SolverConfiguration}, \hyperref[Report]{Report}

\subsection{Syntax}

\subsubsection{Exported Constants}
None
\subsection{Exported Types}
None
\subsubsection{Exported Access Programs}
\begin{center}
\begin{tabular}{p{5cm} p{5cm} p{2cm} p{2cm}}
\hline
\textbf{Name} & \textbf{In} & \textbf{Out} & \textbf{Exceptions} \\
\hline
savePlatoonConfiguration & PlatoonT, string & - & - \\
\hline
saveBridgeConfiguration & BridgeT, string & - & - \\
\hline
saveSolverConfiguration & SolverT, string & - & - \\
\hline
saveReport & ReportT, string & - & - \\
\hline
\end{tabular}
\end{center}

\subsection{Semantics}

\subsubsection{State Variables}
None
\subsubsection{Environment Variables}
None
\subsubsection{Assumptions}
None
\subsubsection{Access Routine Semantics}

\noindent savePlatoonConfiguration(platoon, filepath)
\begin{itemize}
\item output: open outputFile at filepath for write and write each field of platoon to outputFile delineated by ", " in the following order: 
platoon.axleLoad as a string of space separated numbers, platoon.axleSpacing as a string of space separated numbers, number of trucks, headway
Close outputFile.
\item exception: none
\end{itemize}

\noindent saveBridgeConfiguration(bridge, filepath)
\begin{itemize}
\item output: open outputFile at filepath for write and write each field of bridge to outputFile delineated by ", " in the following order: 
bridge.numberSpans, bridge.spanLength as a string of space separated numbers, bridge.concernedSection, bridge.discretizationLength
Close outputFile.
\item exception: none
\end{itemize}

\noindent saveSolverConfiguration(solver, filepath)
\begin{itemize}
\item output: open outputFile at filepath for write and write each field of platoon to outputFile delineated by ", " in the following order: 
solver.forceType, solver.solverType
Close outputFile.
\item exception: none
\end{itemize}

\noindent saveReport(report, filepath)
\begin{itemize}
\item output: open outputFile at path for write. Write a header for the truck platoon configuration in the form "[Platoon]", then write all fields of truck platoon in the form of "field.name = field.value". Do the same for bridge and solver configurations, with their respective headers and fields. Then, write a header for the analysis in the form "[Results]". If the solver was configured for critical sections, write "critical section = value". Proceed to write out the pairs of first axle position and force, separated by commas and each pair separated by new lines. Close the output file.
\item exception: none
\end{itemize}
\subsubsection{Local Functions}
None
\newpage

\section{MIS of Load From File System}

\subsection{Module}\label{Loader} 

Loader

\subsection{Uses}
\hyperref[PlatoonConfiguration]{PlatoonConfiguration}, \hyperref[BridgeConfiguration]{BridgeConfiguration}, \hyperref[SolverConfiguration]{SolverConfiguration}

\subsection{Syntax}

\subsubsection{Exported Constants}
None
\subsection{Exported Types}
None
\subsubsection{Exported Access Programs}
\begin{center}
\begin{tabular}{p{5cm} p{3cm} p{3cm} p{1cm}}
\hline
\textbf{Name} & \textbf{In} & \textbf{Out} & \textbf{Exceptions} \\
\hline
loadPlatoonConfiguration & string & PlatoonT & - \\
\hline
loadBridgeConfiguration & string & BridgeT & - \\
\hline
loadReport & string & \{ PlatoonT, BridgeT, SolverT \} & - \\
\hline
\end{tabular}
\end{center}

\subsection{Semantics}

\subsubsection{State Variables}
None
\subsubsection{Environment Variables}
None
\subsubsection{Assumptions}
None
\subsubsection{Access Routine Semantics}

\noindent loadPlatoonConfiguration(filepath)
\begin{itemize}
\item output: open outputFile at filepath for read and read each field: 
Parse space separated $\mathbb{R}$ until comma as axleLoad, parse next set of space separated $\mathbb{R}$ until the next comma as axleSpacing, parse next $\mathbb{N}$ as numberOfTrucks, parse last $\mathbb{R}$ as headway.
Close outputFile.\\
out := PlatoonT constructed from the parsed values above.
\item exception: none
\end{itemize}

\noindent loadBridgeConfiguration(filepath)
\begin{itemize}
\item output: open outputFile at filepath for read and read each field: 
Parse first number as numberOfSpans, parse space separated $\mathbb{R}$ until comma as spanLength, parse next $\mathbb{R}$ as concernedSection, parse last $\mathbb{R}$ as discretization length.
Close outputFile.\\
out := BridgeT constructed from the parsed values above.
\item exception: none
\end{itemize}

\noindent loadReport(filepath)
\begin{itemize}
\item output: open outputFile at filepath for read. Process lines until the line "[Platoon]" is found. Parse the space-separated numbers on the line containing "axleLoad" as the PlatoonT axleLoad. Parse for axleSpacing in the same manner. Parse for headway and number of trucks in the same manner. Parse for BridgeT in the same manner beginning at "[Bridge]." Parse for SolverT in the same manner beginning at "[Solver]." Close outputFile.\\
out := \{ PlatoonT, BridgeT, SolverT \} constructed from the parsed values above.
\item exception: none
\end{itemize}
\subsubsection{Local Functions}
None
\newpage


\section{MIS of Calculation Call} \label{CalculationCaller}

\subsection{Module}

CalculationCaller

\subsection{Uses}
\hyperref[PlatoonConfiguration]{PlatoonConfiguration}, \hyperref[BridgeConfiguration]{BridgeConfiguration}, \hyperref[SolverConfiguration]{SolverConfiguration}, \hyperref[Report]{Report}

\subsection{Syntax}

\subsubsection{Exported Constants}
None
\subsection{Exported Types}
\noindent \textbf{CalculationInputT} = \{\\
    \begin{tabular}{l}
    truckConfig: TruckT,\\
    bridgeConfig: BridgeT,\\
    solverConfig: SolverT
    \end{tabular}
\\\}

\noindent \textbf{CalculationOutputT} = \{\\
    \begin{tabular}{l}
    \noindent allForces: sequence of sequence of $\mathbb{R}$,\\

	\noindent firstAxlePosition: sequence of $\mathbb{R}$,\\

	\noindent forceConcernedSection: sequence of $\mathbb{R}$,\\
	\noindent forceCriticalSection: sequence of $\mathbb{R}$,\\

	\noindent maxForce: $\mathbb{R}$,\\
	\noindent firstAxlePositionMaxForce: $\mathbb{R}$,\\

    \noindent sections: sequence of $\mathbb{R}$,\\
    \noindent criticalSection: $\mathbb{R}$,\\

	\noindent forceEnvelope: sequence of $\mathbb{R}$,\\
	\noindent firstAxlePositionForceEnvelope: sequence of $\mathbb{R}$
 \end{tabular}
\\\}

\subsubsection{Exported Access Programs}
\begin{center}
\begin{tabular}{p{3cm} p{4cm} p{4cm} p{3cm}}
\hline
\textbf{Name} & \textbf{In} & \textbf{Out} & \textbf{Exceptions} \\
\hline
runCalculation & CalculationInputT & CalculationOutputT & - \\
\hline
\end{tabular}
\end{center}

\subsection{Semantics}

\subsubsection{State Variables}
None
\subsubsection{Environment Variables}
engine: An instance of the MATLAB engine which exports functions that run the relevant MATLAB code.
\subsubsection{Assumptions}
None
\subsubsection{Access Routine Semantics}

\noindent runCalculation(in):
\begin{itemize}
    \item output: $out := \\
        (in.solverConfig.solverType = CONCERNED \Rightarrow\\ concernedSection(in)\ | \\
        in.solverConfig.solverType = CRITICAL \Rightarrow\\ criticalSection(in)) \land\\
        report.updateReport(ReportT \{ in, out \})$
    \item exception: none
\end{itemize}

\subsubsection{Local Functions}
\noindent \textbf{concernedSection}: $CalculationInputT \rightarrow CalculationOutputT$:
\begin{itemize}
    \item output: $out :=\\
	    (in.bridgeConfig.numberSpans = 1 \Rightarrow\\ concernedSectionOneSpan(in)\ | \\
        in.bridgeConfig.numberSpans = 2 \Rightarrow\\ concernedSectionTwoSpan(in)\ | \\
        in.bridgeConfig.numberSpans = 3 \Rightarrow\\ concernedSectionThreeSpan(in))$
    \item exception: none
\end{itemize}

\noindent \textbf{concernedSectionOneSpan}: $CalculationInputT \rightarrow CalculationOutputT$:
\begin{itemize}
    \item output: $out :=\\
	    (in.solverConfig.forceType = POSITIVE\_MOMENT \Rightarrow\\ concernedSectionOneSpanMoment(in)\ | \\
	    in.solverConfig.forceType = SHEAR \Rightarrow\\ concernedSectionOneSpanShear(in))$
    \item exception: none 
\end{itemize}

\noindent \textbf{concernedSectionOneSpanMoment}: $CalculationInputT \rightarrow CalculationOutputT$:
\begin{itemize}
    \item output: $out := engine.Concerned\_section\_one\_span\_moment(in)$
    \item exception: none 
\end{itemize}
    
\noindent \textbf{concernedSectionOneSpanShear}: $CalculationInputT \rightarrow CalculationOutputT$
\begin{itemize}
    \item output: $out := engine.Concerned\_section\_one\_span\_shear(in)$
    \item exception: none
\end{itemize}

\noindent \textbf{concernedSectionTwoSpan}: $CalculationInputT \rightarrow CalculationOutputT$
\begin{itemize}
    \item output: $out :=\\
	(in.solverConfig.forceType = POSITIVE\_MOMENT \Rightarrow\\ concernedSectionTwoSpanPositiveMoment(in) \ | \\
    in.solverConfig.forceType = NEGATIVE\_MOMENT \Rightarrow\\ concernedSectionTwoSpanNegativeMoment(in) \ | \\
	in.solverConfig.forceType = SHEAR \Rightarrow\\ concernedSectionTwoSpanShear(in))$
    \item exception: none
\end{itemize}

\noindent \textbf{concernedSectionTwoSpanPositiveMoment}: $CalculationInputT \rightarrow CalculationOutputT$
\begin{itemize}
    \item output: $out := engine.Concerned\_section\_two\_span\_moment(in)$
    \item exception: none
\end{itemize}

\noindent \textbf{concernedSectionTwoSpanNegativeMoment}: $CalculationInputT \rightarrow CalculationOutputT$
\begin{itemize}
    \item output: $out := engine.Concerned\_section\_two\_span\_moment(in)$
    \item exception: none
\end{itemize}

\noindent \textbf{concernedSectionTwoSpanShear}: $CalculationInputT \rightarrow CalculationOutputT$
\begin{itemize}
    \item output: $out := engine.Concerned\_section\_two\_span\_shear(in)$
    \item exception: none
\end{itemize}

\noindent \textbf{concernedSectionThreeSpan}: $CalculationInputT \rightarrow CalculationOutputT$
\begin{itemize}
    \item output: $out :=\\
        (in.solverConfig.forceType = POSITIVE\_MOMENT \Rightarrow\\ concernedSectionThreeSpanPositiveMoment(in)\ | \\
        in.solverConfig.forceType = NEGATIVE\_MOMENT \Rightarrow\\ concernedSectionThreeSpanNegativeMoment(in)\ | \\
        in.solverConfig.forceType = SHEAR  \Rightarrow\\ concernedSectionThreeSpanShear(in))$
    \item exception: none
\end{itemize}

\noindent \textbf{concernedSectionThreeSpanPositiveMoment}: $CalculationInputT \rightarrow CalculationOutputT$
\begin{itemize}
    \item output: $out := engine.Concerned\_section\_three\_span\_moment(in)$
    \item exception: none
\end{itemize}

\noindent \textbf{concernedSectionThreeSpanNegativeMoment}: $CalculationInputT \rightarrow CalculationOutputT$
\begin{itemize}
    \item output: $out := engine.Concerned\_section\_three\_span\_moment(in)$
    \item exception: none
\end{itemize}

\noindent \textbf{concernedSectionThreeSpanShear}: $CalculationInputT \rightarrow CalculationOutputT$
\begin{itemize}
    \item output: $out := engine.Concerned\_section\_three\_span\_shear(in)$
    \item exception: none
\end{itemize}

\noindent \textbf{criticalSection}: $CalculationInputT \rightarrow CalculationOutputT$
\begin{itemize}
    \item output: $out :=\\
	  (in.bridgeConfig.numberSpans = 1 \Rightarrow\\ criticalSectionOneSpan(in)\ | \\
        in.bridgeConfig.numberSpans = 2 \Rightarrow\\ criticalSectionTwoSpan(in)\ | \\
        in.bridgeConfig.numberSpans = 3 \Rightarrow\\ criticalSectionThreeSpan(in))$
    \item exception: none
\end{itemize} 

\noindent \textbf{criticalSectionOneSpan}: $CalculationInputT \rightarrow CalculationOutputT$
\begin{itemize}
    \item output: $out :=\\
	  (in.solverConfig.forceType = POSITIVE\_MOMENT \Rightarrow\\ criticalSectionOneSpanMoment(in)\ | \\
	  in.solverConfig.forceType = SHEAR \Rightarrow\\ criticalSectionOneSpanShear(in))$
    \item exception: none
\end{itemize}

\noindent \textbf{criticalSectionOneSpanMoment}: $CalculationInputT \rightarrow CalculationOutputT$
\begin{itemize}
    \item output: $out := engine.Critical\_section\_one\_span\_moment(in)$
    \item exception: none
\end{itemize}

\noindent \textbf{criticalSectionOneSpanShear}: $CalculationInputT \rightarrow CalculationOutputT$
\begin{itemize}
    \item output: $out := Critical\_section\_one\_span\_shear(in)$
    \item exception: none
\end{itemize}

\noindent \textbf{criticalSectionTwoSpan}: $CalculationInputT \rightarrow CalculationOutputT$
\begin{itemize}
    \item output: $out :=\\
	(in.solverConfig.forceType = POSITIVE\_MOMENT) \Rightarrow\\ criticalSectionTwoSpanPositiveMoment(in)\ | \\
    in.solverConfig.forceType = NEGATIVE\_MOMENT) \Rightarrow\\ criticalSectionTwoSpanNegativeMoment(in)\ | \\
	in.solverConfig.forceType = SHEAR \Rightarrow\\ criticalSectionTwoSpanShear(in))$
    \item exception: none
\end{itemize}

\noindent \textbf{criticalSectionTwoSpanPositiveMoment}: $CalculationInputT \rightarrow CalculationOutputT$
\begin{itemize}
    \item output: $out := engine.Critical\_section\_two\_span\_positive\_moment(in)$
    \item exception: none
\end{itemize}

\noindent \textbf{criticalSectionTwoSpanNegativeMoment}: $CalculationInputT \rightarrow CalculationOutputT$
\begin{itemize}
    \item output: $out := engine.Critical\_section\_two\_span\_negative\_moment(in)$
    \item exception: none
\end{itemize}

\noindent \textbf{criticalSectionTwoSpanShear}: $CalculationInputT \rightarrow CalculationOutputT$
\begin{itemize}
    \item output: $out := engine.Critical\_section\_two\_span\_shear(in)$
    \item exception: none
\end{itemize}

\noindent \textbf{criticalSectionThreeSpan}: $CalculationInputT \rightarrow CalculationOutputT$
\begin{itemize}
    \item output: $out := \\
	  (in.solverConfig.forceType = POSITIVE\_MOMENT \Rightarrow\\ criticalSectionThreeSpanPositiveMoment(in)\ | \\
        in.solverConfig.forceType = NEGATIVE\_MOMENT) \Rightarrow\\ criticalSectionThreeSpanNegativeMoment(in)\ | \\
	  in.solverConfig.forceType = SHEAR \Rightarrow\\ criticalSectionThreeSpanShear(in))$
    \item exception: none
\end{itemize}

\noindent \textbf{criticalSectionThreeSpanPositiveMoment}: $CalculationInputT \rightarrow CalculationOutputT$
\begin{itemize}
    \item output: $out := engine.Critical\_section\_three\_span\_positive\_moment(in)$
    \item exception: none
\end{itemize}

\noindent \textbf{criticalSectionThreeSpanNegativeMoment}: $CalculationInputT \rightarrow CalculationOutputT$
\begin{itemize}
    \item output: $out := engine.Critical\_section\_three\_span\_negative\_moment(in)$
    \item exception: none
\end{itemize}

\noindent \textbf{criticalSectionThreeSpanShea}: $CalculationInputT \rightarrow CalculationOutputT$
\begin{itemize}
    \item output: $out := engine.Critical\_section\_three\_span\_shear(in)$
    \item exception: none
\end{itemize}

\newpage

\section{MIS of Truck Platoon Configuration Data Format} \label{PlatoonConfiguration} 

\subsection{Module}

PlatoonConfiguration

\subsection{Uses}
None

\subsection{Syntax}

\subsubsection{Exported Constants}
None
\subsection{Exported Types}
PlatoonT = \{\\
    \begin{tabular}{l}
	axleLoad: sequence of $\mathbb{R}$,\\
	axleSpacing: sequence of $\mathbb{R}$,\\
	numberOfTrucks: $\mathbb{N}$,\\
    headway: $\mathbb{R}$
    \end{tabular}
\\\}
\subsubsection{Exported Access Programs}
\begin{center}
\begin{tabular}{p{5cm} p{2cm} p{4cm} p{5cm}}
\hline
\textbf{Name} & \textbf{In} & \textbf{Out} & \textbf{Exceptions} \\
\hline
new PlatoonConfiguration & - & PlatoonConfiguration & - \\
\hline
getConfiguration & - & PlatoonT & -\\
\hline
updateAxleLoad & seq of $\mathbb{R}$ & - & -\\
\hline
updateAxleSpacing & seq of $\mathbb{R}$ & - & -\\
\hline
updateNumberOfTrucks & $\mathbb{N}$ & - & -\\
\hline
updateHeadway & $\mathbb{R}$	 & - & -\\
\end{tabular}
\end{center}

\subsection{Semantics}

\subsubsection{State Variables}
axleLoad: sequence of $\mathbb{R}$,\\
axleSpacing: sequence of $\mathbb{R}$,\\
numberOfTrucks: $\mathbb{N}$,\\
headway: $\mathbb{R}$
\subsubsection{Environment Variables}
None
\subsubsection{Assumptions}
None
\subsubsection{Access Routine Semantics}

\noindent new PlatoonConfiguration():
\begin{itemize}
\item transition:\\
        \begin{tabular}{p{1cm} p{8cm}}
		& axleLoad := $\emptyset$\\
		& axleSpacing := $\emptyset$\\
		& numberOfTrucks := 0\\
		& headway := 0.0\\
        \end{tabular}
\item output: out := self
\item exception: none
\end{itemize}

\noindent PlatoonT getConfiguration():
\begin{itemize}
\item output: out := PlatoonT \{ axleLoad, axleSpacing, numberOfTrucks, headway \} 
\item exception: none
\end{itemize}

\noindent updateAxleLoad(newAxleLoad):
\begin{itemize}
\item transition: self.axleLoad := newAxleLoad
\item exception: none
\end{itemize}

\noindent updateAxleSpacing(newAxleSpacing):
\begin{itemize}
\item transition: self.axleSpacing := newAxleSpacing
\item exception: none
\end{itemize}

\noindent updateNumberOfTrucks(newNumberOfTrucks)
\begin{itemize}
\item transition: self.numberOfTrucks := newNumberOfTrucks
\item exception: none
\end{itemize}

\noindent updateHeadway(newHeadway)
\begin{itemize}
\item transition: self.headway := newHeadway
\item exception: none
\end{itemize}

\subsubsection{Local Functions}
None
\newpage

\section{MIS of Bridge Configuration Data Format} \label{BridgeConfiguration} 

\subsection{Module}

BridgeConfiguration

\subsection{Uses}
None

\subsection{Syntax}

\subsubsection{Exported Constants}
None
\subsection{Exported Types}
BridgeT = \{\\
    \begin{tabular}{l}
	numberSpans: \{1, 2, 3\},\\
	spanLength: seq of $\mathbb{R}$,\\
	concernedSection: $\mathbb{R}$,\\
    discretizationLength: $\mathbb{R}$,\\
    \end{tabular}
\\\}
\subsubsection{Exported Access Programs}
\begin{center}
\begin{tabular}{p{5cm} p{2cm} p{4cm} p{4cm}}
\hline
\textbf{Name} & \textbf{In} & \textbf{Out} & \textbf{Exceptions} \\
\hline
new BridgeConfiguration & - & BridgeConfiguration & - \\
\hline
getConfiguration & - & BridgeT & -\\
\hline
updateNumberOfSpans & string & - & invalidConfigurationValue \\
\hline

\hline
updateSpanLength & string & - & invalidConfigurationValue \\
\hline

\hline
updateConcernedSection & string & - & invalidConfigurationValue \\
\hline

\hline
updateDiscretizationLength & string & - & invalidConfigurationValue \\
\hline

\end{tabular}
\end{center}

\subsection{Semantics}

\subsubsection{State Variables}
numberOfSpans: $\mathbb{N}$\\
spanLength: seq of $\mathbb{R}$ \\
concernedSection: $\mathbb{R}$\\
discretizationLength: $\mathbb{R}$\\

\subsubsection{Environment Variables}
None
\subsubsection{Assumptions}
None
\subsubsection{Access Routine Semantics}

\noindent new BridgeConfiguration():
\begin{itemize}
\item transition:\\
        \begin{tabular}{p{1cm} p{8cm}}
		& numberOfSpans := 0\\
		& spanLength := empty sequence\\
		& concernedSection := 0.0\\
		& discretizationLength := 0.0\\
        \end{tabular}
\item output: out := self
\item exception: none
\end{itemize}

\noindent BridgeT getConfiguration():
\begin{itemize}
\item output: out := BridgeT \{ numberOfSpans, spanLength, concernedSection, discretizationLength \}
\item exception: none
\end{itemize}

\noindent updateNumberOfSpans(newNumberOfSpans):
\begin{itemize}
\item transition: Parse newNumberOfSpans as $\mathbb{N}$. Set self.numberOfSpans to the parsed value of newNumberOfSpans.
\item exception: $exc := newNumberOfSpans \notin \mathbb{N} \Rightarrow invalidConfigurationValue$
\end{itemize}

\noindent updateSpanLength(newSpanLength):
\begin{itemize}
\item transition: Parse newSpanLength as a space-separated sequence of $\mathbb{R}$. Set self.axleLoad to the parsed value of newAxleLoad.
\item exception: $exc := \exists(n | n \in newSpanLength \land (n \notin \mathbb{N} \land n \neq \text{' '})) \Rightarrow invalidConfigurationValue$
\end{itemize}

\noindent updateConcernedSection(newConcernedSection)
\begin{itemize}
\item transition: Parse newConcernedSection as $\mathbb{R}$. Set self.concernedSection to the parsed value of newConcernedSection.
\item exception: $exc := newConcernedSection \notin \mathbb{R} \Rightarrow invalidConfigurationValue$
\end{itemize}

\noindent updateDiscretizationLength(newDiscretizationLength)
\begin{itemize}
\item transition: Parse newDiscretizationLength as $\mathbb{R}$. Set self.discretizationLength to the parsed value of newDiscretizationLength.
\item exception: $exc := newDiscretizationLength \notin \mathbb{R} \Rightarrow invalidConfigurationValue$
\end{itemize}

\subsubsection{Local Functions}
None
\newpage

\section{MIS of Solver Configuration Data Format} \label{SolverConfiguration} 

\subsection{Module}

SolverConfiguration

\subsection{Uses}
None

\subsection{Syntax}

\subsubsection{Exported Constants}
None
\subsection{Exported Types}
SolverT = \{\\
    \begin{tabular}{l}
	forceType: \{POSITIVE\_MOMENT, NEGATIVE\_MOMENT,
		SHEAR\},\\
    solverType: \{CONCERNED,
		CRITICAL\}\\
    \end{tabular}
\\\}
\subsubsection{Exported Access Programs}
\begin{center}
\begin{tabular}{p{5cm} p{2cm} p{4cm} p{4cm}}
\hline
\textbf{Name} & \textbf{In} & \textbf{Out} & \textbf{Exceptions} \\
\hline
new SolverConfiguration & - & SolverConfiguration & - \\
\hline
getConfiguration & - & SolverT & - \\
\hline
updateForceType & string & - & invalidConfigurationValue \\
\hline

\hline
updateSolverType & string & - & invalidConfigurationValue \\
\hline

\end{tabular}
\end{center}

\subsection{Semantics}

\subsubsection{State Variables}
forceType: SolverT.forceType\\
solverType: SolverT.solverType

\subsubsection{Environment Variables}
None
\subsubsection{Assumptions}
None
\subsubsection{Access Routine Semantics}

\noindent new SolverConfiguration():
\begin{itemize}
\item transition:\\
        \begin{tabular}{p{1cm} p{8cm}}
        & forceType := POSITIVE\_MOMENT\\
	& solverType := CONCERNED
        \end{tabular}
\item output: out := self
\item exception: none
\end{itemize}

\noindent getConfiguration():
\begin{itemize}
\item transition: none
\item output: out := SolverT \{ self.forceType, self.solverType\}
\item exception:none
\end{itemize}

\noindent updateForceType(newForceType):
\begin{itemize}
\item transition: \\
\begin{tabular}{p{1cm}p{16cm}}
& (newForceType = "Positive Moment" $\Rightarrow$ forceType := POSITIVE\_MOMENT $|$ \\
& newForceType = "Negative Moment" $\Rightarrow$ forceType := NEGATIVE\_MOMENT $|$\\
& newForceType = "Shear" $\Rightarrow$ forceType := SHEAR)\\
\end{tabular}
\item exception: $exc := newForceType \notin \{\text{"Positive Moment", "Negative Moment", "Shear"}\} \Rightarrow invalidConfigurationValue$
\end{itemize}

\noindent updateSolverType(newSolverType):
\begin{itemize}
\item transition:\\
\begin{tabular}{p{1cm} p{16cm}}
& (newSolverType = "Concerned Section" $\Rightarrow$ solverType := CONCERNED $|$\\
& newSolverType = "Negative Moment" $\Rightarrow$ solverType := CRITICAL)\\
\end{tabular}
\item exception: $exc := newSolverType \notin \{\text{"Concerned Section", "Negative Moment"}\} \Rightarrow invalidConfigurationValue$
\end{itemize}

\subsubsection{Local Functions}
None
\newpage

\section{MIS of Report Data Format} \label{Report} 

\subsection{Module}

Report

\subsection{Uses}
None

\subsection{Syntax}

\subsubsection{Exported Constants}
None
\subsection{Exported Types}
ReportT = \{\\
    \begin{tabular}{l}
	input: CalculationInputT\\
    results: CalculationOutputT\\
    \end{tabular}
\\\}
\subsubsection{Exported Access Programs}
\begin{center}
\begin{tabular}{p{5cm} p{2cm} p{4cm} p{4cm}}
\hline
\textbf{Name} & \textbf{In} & \textbf{Out} & \textbf{Exceptions} \\
\hline
new Report & - & Report & - \\
\hline
getReport & - & ReportT & invalidReport \\
\hline
updateReport & ReportT & - & - \\
\hline

\end{tabular}
\end{center}

\subsection{Semantics}

\subsubsection{State Variables}
report: ReportT

\subsubsection{Environment Variables}
None
\subsubsection{Assumptions}
None
\subsubsection{Access Routine Semantics}

\noindent new Report():
\begin{itemize}
\item transition: report := NULL
\item output: out := self
\item exception: none
\end{itemize}

\noindent getReport():
\begin{itemize}
\item output: out := report
\item exception: report = NULL $\Rightarrow$ invalidReport
\end{itemize}

\noindent updateReport(ReportT newReport):
\begin{itemize}
\item transition: report := newReport
\item exception: none
\end{itemize}

\subsubsection{Local Functions}
None
\newpage

\section{MIS of Truck Platoon Configuration Save To File System Visualization} \label{TruckSaver}

\subsection{Module}

TruckSaver

\subsection{Uses}
\hyperref[Saver]{Saver}, \hyperref[PlatoonConfiguration]{PlatoonConfiguration}

\subsection{Syntax}

\subsubsection{Exported Constants}
None
\subsection{Exported Types}
None

\subsubsection{Exported Access Programs}
\begin{center}
\begin{tabular}{p{5cm} p{2cm} p{4cm} p{5.5cm}}
\hline
\textbf{Name} & \textbf{In} & \textbf{Out} & \textbf{Exceptions} \\
\hline
new TruckSaver & - & TruckSaver & - \\
\hline
buttonPressed & - & - & - \\
\hline
\end{tabular}
\end{center}

\subsection{Semantics}

\subsubsection{State Variables}
None
\subsubsection{Environment Variables}
None
\subsubsection{Assumptions}
None
\subsubsection{Access Routine Semantics}

\noindent {new TruckSaver}()
\begin{itemize}
\item output: out := TruckSaver
\item exception: none
\end{itemize}

\noindent  {buttonPressed}()
\begin{itemize}
\item output: When the button is clicked, open a windows file dialog to select where to save the file. After the user selects a valid location and clicks save, the method shall call PlatoonConfiguration().getConfiguration() to get the current truck configuration. Then, the method shall call savePlatoonConfiguration with the parsed configuration and the filepath the user specified.
\item exception: none
\end{itemize}

\subsubsection{Local Functions}
None
\newpage
\section{MIS of Bridge Configuration Save To File System Visualization} \label{BridgeSaver}

\subsection{Module}

BridgeSaver

\subsection{Uses}
\hyperref[Saver]{Saver}, \hyperref[BridgeConfiguration]{BridgeConfiguration}

\subsection{Syntax}

\subsubsection{Exported Constants}
None
\subsection{Exported Types}
None

\subsubsection{Exported Access Programs}
\begin{center}
\begin{tabular}{p{5cm} p{2cm} p{4cm} p{5.5cm}}
\hline
\textbf{Name} & \textbf{In} & \textbf{Out} & \textbf{Exceptions} \\
\hline
new BridgeSaver & - & BridgeSaver & - \\
\hline
buttonPressed & - & - & - \\
\hline
\end{tabular}
\end{center}

\subsection{Semantics}

\subsubsection{State Variables}
None
\subsubsection{Environment Variables}
None
\subsubsection{Assumptions}
None
\subsubsection{Access Routine Semantics}

\noindent {new BridgeSaver}()
\begin{itemize}
\item output: out :=  BridgeSaver
\item exception: none
\end{itemize}

\noindent  {buttonPressed}()
\begin{itemize} 
\item output: When the button is clicked, open a windows file dialog to select where to save the file. After the user selects a valid location and clicks save, the method shall call BridgeConfiguration().getConfiguration() to get the current bridge configuration. Then, the method shall call saveBridgeConfiguration with the parsed configuration and the filepath the user specified.
\item exception: none
\end{itemize}

\subsubsection{Local Functions}
None
\newpage
\section{MIS of Solver Configuration Save To File System Visualization} \label{SolverSaver}

\subsection{Module}

SolverSaver

\subsection{Uses}
\hyperref[Saver]{Saver}, \hyperref[SolverConfiguration]{SolverConfiguration}, \hyperref[Report]{Report}

\subsection{Syntax}

\subsubsection{Exported Constants}
None
\subsection{Exported Types}
None

\subsubsection{Exported Access Programs}
\begin{center}
\begin{tabular}{p{5cm} p{2cm} p{4cm} p{5.5cm}}
\hline
\textbf{Name} & \textbf{In} & \textbf{Out} & \textbf{Exceptions} \\
\hline
new SolverSaver & - & SolverSaver & - \\
\hline
buttonPressed & - & - & - \\
\hline
\end{tabular}
\end{center}

\subsection{Semantics}

\subsubsection{State Variables}
None
\subsubsection{Environment Variables}
None
\subsubsection{Assumptions}
None
\subsubsection{Access Routine Semantics}

\noindent {new SolverSaver}()
\begin{itemize}  
\item output: out :=  SolverSaver
\item exception: none
\end{itemize}

\noindent  {buttonPressed}()
\begin{itemize}
\item output: When the button is clicked, open a windows file dialog to select where to save the file. After the user selects a valid location and clicks save, the method shall call Report.getReport() to get the current results, plot them, and then save the graph to the filepath the user specified.
\item exception: none
\end{itemize}

\subsubsection{Local Functions}
None
\newpage
\section{MIS of Output Report Save To File System Visualization} \label{ReportSaver}

\subsection{Module}

ReportSaver

\subsection{Uses}
\hyperref[Saver]{Saver}, \hyperref[Report]{Report}

\subsection{Syntax}

\subsubsection{Exported Constants}
None
\subsection{Exported Types}
None

\subsubsection{Exported Access Programs}
\begin{center}
\begin{tabular}{p{5cm} p{2cm} p{4cm} p{5.5cm}}
\hline
\textbf{Name} & \textbf{In} & \textbf{Out} & \textbf{Exceptions} \\
\hline
new ReportSaver & - & ReportSaver & - \\
\hline
buttonPressed & - & - & - \\
\hline
\end{tabular}
\end{center}

\subsection{Semantics}

\subsubsection{State Variables}
None
\subsubsection{Environment Variables}
None
\subsubsection{Assumptions}
None
\subsubsection{Access Routine Semantics}

\noindent {new ReportSaver}()
\begin{itemize}
\item output: out :=  ReportSaver
\item exception: none
\end{itemize}

\noindent  {buttonPressed}()
\begin{itemize}
\item output: When the button is clicked, open a windows file dialog to select where to save the file. After the user selects a valid location and clicks save, the method shall get the most recently generated output report with Report().getReport(). Then, the method shall call saveReport with the parsed report and the filepath the user specified.
\item exception: none
\end{itemize}

\subsubsection{Local Functions}
None

\newpage
\section{MIS of Truck Platoon Configuration Load From File System Visualization} \label{TruckLoader}

\subsection{Module}

TruckLoader

\subsection{Uses}
\hyperref[Loader]{Loader}, \hyperref[PlatoonConfiguration]{PlatoonConfiguration}

\subsection{Syntax}

\subsubsection{Exported Constants}
None
\subsection{Exported Types}
None

\subsubsection{Exported Access Programs}
\begin{center}
\begin{tabular}{p{5cm} p{2cm} p{4cm} p{5.5cm}}
\hline
\textbf{Name} & \textbf{In} & \textbf{Out} & \textbf{Exceptions} \\
\hline
new TruckLoader & - & TruckLoader & - \\
\hline
buttonPressed & - & - & - \\
\hline
\end{tabular}
\end{center}

\subsection{Semantics}

\subsubsection{State Variables}
None
\subsubsection{Environment Variables}
None
\subsubsection{Assumptions}
None
\subsubsection{Access Routine Semantics}

\noindent {new ReportSaver}()
\begin{itemize}
\item output: out :=  TruckLoader
\item exception: none
\end{itemize}

\noindent  {buttonPressed}()
\begin{itemize} 
\item output: When the button is clicked, open a windows file dialog to select the file containing the platoon configuration. After the user selects a file and clicks load, the method shall call loadPlatoonConfiguration() with the file specified by the user. Then, the method shall set the values of TruckConfiguration() with the respective values.
\item exception: none
\end{itemize}

\subsubsection{Local Functions}
None
\newpage
\section{MIS of Bridge Configuration Load From File System Visualization} \label{BridgeLoader}

\subsection{Module}

BridgeLoader

\subsection{Uses}
\hyperref[Loader]{Loader}, \hyperref[BridgeConfiguration]{BridgeConfiguration}

\subsection{Syntax}

\subsubsection{Exported Constants}
None
\subsection{Exported Types}
None

\subsubsection{Exported Access Programs}
\begin{center}
\begin{tabular}{p{5cm} p{2cm} p{4cm} p{5.5cm}}
\hline
\textbf{Name} & \textbf{In} & \textbf{Out} & \textbf{Exceptions} \\
\hline
new BridgeLoader & - & BridgeLoader & - \\
\hline
buttonPressed & - & - & - \\
\hline
\end{tabular}
\end{center}

\subsection{Semantics}

\subsubsection{State Variables}
None
\subsubsection{Environment Variables}
None
\subsubsection{Assumptions}
None
\subsubsection{Access Routine Semantics}

\noindent {new BridgeLoader}()
\begin{itemize}
\item output: out :=  BridgeLoader
\item exception: none
\end{itemize}

\noindent  {buttonPressed}()
\begin{itemize}
\item output: When the button is clicked, open a windows file dialog to select the file containing the bridge configuration. After the user selects a file and clicks load, the method shall call loadBridgeConfiguration() with the file specified by the user. Then, the method shall set the values of BridgeConfiguration() with the respective values.
\item exception: none
\end{itemize}

\subsubsection{Local Functions}
None
\newpage
\section{MIS of Output Report Load From File System Visualization} \label{ReportLoader}

\subsection{Module}

ReportLoader

\subsection{Uses}
\hyperref[Loader]{Loader}, \hyperref[Report]{Report}

\subsection{Syntax}

\subsubsection{Exported Constants}
None
\subsection{Exported Types}
None

\subsubsection{Exported Access Programs}
\begin{center}
\begin{tabular}{p{5cm} p{2cm} p{4cm} p{5.5cm}}
\hline
\textbf{Name} & \textbf{In} & \textbf{Out} & \textbf{Exceptions} \\
\hline
new ReportLoader & - & ReportLoader & - \\
\hline
buttonPressed & - & - & - \\
\hline
\end{tabular}
\end{center}

\subsection{Semantics}

\subsubsection{State Variables}
None
\subsubsection{Environment Variables}
None
\subsubsection{Assumptions}
None
\subsubsection{Access Routine Semantics}

\noindent {new ReportLoader}()
\begin{itemize}
\item output: out := ReportLoader
\item exception: none
\end{itemize}

\noindent  {buttonPressed}()
\begin{itemize}
\item output: When the button is clicked, open a windows file dialog to select the file containing the report. After the user selects a file and clicks load, the method shall call loadReport() with the file specified by the user. Then, the method shall set the values of PlatoonConfiguration, BridgeConfiguration, and SolverConfiguration with the respective values.
\item exception: none
\end{itemize}

\subsubsection{Local Functions}
None
%*************************************************
%*************************************************
%*************************************************
%*************************************************
%*************************************************
\newpage

\newpage
\section{MIS of Truck Platoon Configuration Visualization} \label{PlatoonVisualizer}

\subsection{Module}

PlatoonVisualizer

\subsection{Uses}
\hyperref[PlatoonConfiguration]{PlatoonConfiguration}

\subsection{Syntax}

\subsubsection{Exported Constants}
None
\subsection{Exported Types}
None

\subsubsection{Exported Access Programs}
\begin{center}
\begin{tabular}{p{5cm} p{2cm} p{4cm} p{5.5cm}}
\hline
\textbf{Name} & \textbf{In} & \textbf{Out} & \textbf{Exceptions} \\
\hline
new PlatoonVisualizer & - & PlatoonVisualizer & - \\
\hline
numAxlesChanged & $\mathbb{N}$ & - & -\\
\hline
platoonConfigured & - & - & -\\
\hline
visualize & - & - & -\\
\hline
\end{tabular}
\end{center}

\subsection{Semantics}

\subsubsection{State Variables}
numAxles: $\mathbb{N}$\\
axleLoad: sequence of $\mathbb{R}$\\
axleSpacing: sequence of $\mathbb{R}$\\
numberOfTrucks: $\mathbb{N}$\\
headway: $\mathbb{R}$
\subsubsection{State Invariants}
$numAxles \geq 3 \land numAxles \leq 17$\\
$|axleLoad| \leq numberOfTrucks * numAxles$\\
$|axleSpacing| \leq numberOfTrucks * (numAxles - 1)$\\
$\forall(x : \mathbb{R} | x \in axleLoad \land x \geq 0.1 \land x \leq 1000)$\\
$\forall(x : \mathbb{R} | x \in axleSpacing \land x \geq 1.2 \land x \leq 20)$\\
$numberOfTrucks > 0$\\
$headway \geq 5.0$
\subsubsection{Environment Variables}
None
\subsubsection{Assumptions}
None
\subsubsection{Access Routine Semantics}

\noindent \textbf{new PlatoonVisualizer}()
\begin{itemize}
\item transition: \\
		numAxles := 3\\
		axleLoad := $\emptyset$\\
		axleSpacing := $\emptyset$\\
		numberOfTrucks := 1\\
		headway := 5.0
\item output: out := self
\item exception: none
\end{itemize}

\noindent \textbf{numAxlesChanged}(i)
\begin{itemize}
\item transition: numAxles := $i$. Add or remove input fields until there are $i$ axle load fields and $i-1$ axle spacing fields. Call platoonConfigured.
\item exception: none
\end{itemize}

\noindent \textbf{platoonConfigured}()
\begin{itemize}
\item transition: If\\
	\begin{tabular}{p{1cm} p{14cm}}
	& $|axleLoad| = numberOfTrucks * numAxles \land$\\
	& $|axleSpacing| = numberOfTrucks * (numAxles - 1)$
	\end{tabular}
	call\\
	\begin{tabular}{p{1cm} p{14cm}}
	& TruckConfiguration.updateAxleLoad(axleLoad)\\
	& TruckConfiguration.updateAxleSpacing(axleSpacing)\\
	& TruckConfiguration.updateNumberOfTrucks(numberOfTrucks)\\
	& TruckConfiguration.updateHeadway(headway)
	\end{tabular}
\item exception: none
\end{itemize}

\noindent \textbf{visualize}()
\begin{itemize}
\item output: Call TruckConfiguration.getConfig(), and do the following with the result:
\begin{itemize}
\item Draw truck based on truck.axleSpacing.
\item Then add an offset of truck.headway.	
\item Do this truck.numberOfTrucks times.
\end{itemize}
Draw the fields for axleLoad, axleSpacing, numberOfTrucks, headway using the respective state variables as their values.
\item exception: none
\end{itemize}
\subsubsection{Local Functions}
None

\newpage
\section{MIS of Bridge Configuration Visualization} \label{BridgeVisualizer}

\subsection{Module}

BridgeVisualizer

\subsection{Uses}
\hyperref[BridgeConfiguration]{BridgeConfiguration}

\subsection{Syntax}

\subsubsection{Exported Constants}
None
\subsection{Exported Types}
None

\subsubsection{Exported Access Programs}
\begin{center}
\begin{tabular}{p{5cm} p{2cm} p{4cm} p{5.5cm}}
\hline
\textbf{Name} & \textbf{In} & \textbf{Out} & \textbf{Exceptions} \\
\hline
new BridgeVisualizer & - & BridgeVisualizer & -\\
\hline
numberOfSpansDetermined & seq of $\mathbb{R}$ & - & -\\
\hline
bridgeConfigEdited & - & - & -\\
\hline
visualize & - & - & -\\
\hline
\end{tabular}
\end{center}

\subsection{Semantics}

\subsubsection{State Variables}
numberOfSpans: $\mathbb{N}$\\
spanLength: $\mathbb{R}$\\
concernedSection: $\mathbb{R}$\\
discretizationLength: $\mathbb{R}$
\subsubsection{State Invariants}
$numberOfSpans \geq 1 \land numberOfSpans \leq 3$\\
$|spanLength| \leq numberOfSpans$\\
$\forall(x : \mathbb{R} | x \in spanLength \land x \geq 0.0)$\\
$concernedSection \geq 0.0 \land concernedSection \leq spanLength$\\
$discretizationLength \geq 0.0$
\subsubsection{Environment Variables}
None
\subsubsection{Assumptions}
None
\subsubsection{Access Routine Semantics}

\noindent \textbf{new BridgeVisualizer}()
\begin{itemize}
\item transition: \\
		numberOfSpans := 1\\
		spanLength := 0.0\\
		concernedSection := 0.0\\
		discretizationLength := 0.0
\item output: out := self
\item exception: none
\end{itemize}

\noindent \textbf{numberOfSpansDetermined}(numberOfSpans)
\begin{itemize}
\item transition: Whenever the user updates the numberOfSpans field, set self.numberOfSpans := numberOfSpans. Add or remove span length input fields until there are $numberOfSpans$ fields. Call bridgeConfigEdited().
\item exception: none
\end{itemize}

\noindent \textbf{bridgeConfigEdited}()
\begin{itemize}
\item transition: Whenever the user updates an input field, if $|spanLength| = numberOfSpans$, convert numberOfSpans, spanLength, concernedSection, and discretizationLength into strings and call\\
	\begin{tabular}{p{1cm} p{14cm}}
	& BridgeConfiguration.updateNumberOfSpans(numberOfSpans)\\
	& BridgeConfiguration.updateSpanLength(spanLength)\\
	& BridgeConfiguration.updateConcernedSection(concernedSection)\\
	& BridgeConfiguration.updateDiscretizationLength(discretizationLength)
	\end{tabular}
\item exception: none
\end{itemize}

\noindent \textbf{visualize}()
\begin{itemize}
\item output: Call BridgeConfiguration.getConfig(), and do the following with the result:
\begin{itemize}
\item Draw bridge using bridge.numberOfSpans and bridge.spanLength.
\item If bridge.discretizationLength is defined, then draw the separate bridge sections of size bridge.discretizationLength.	
\item If bridge.sectionOfConcern is defined, then show a vertical line at the position of bridge.sectionOfConcern.
\end{itemize}
Draw the input fields for numberOfSpans, spanLength, concernedSection, discretizationLength using the respective state variables as their values.
\item exception: none
\end{itemize}
\subsubsection{Local Functions}
None

\newpage
\section{MIS of Solver Configuration Visualization} \label{SolverVisualizer}

\subsection{Module}

SolverVisualizer

\subsection{Uses}
\hyperref[SolverConfiguration]{SolverConfiguration}

\subsection{Syntax}

\subsubsection{Exported Constants}
None
\subsection{Exported Types}
None

\subsubsection{Exported Access Programs}
\begin{center}
\begin{tabular}{p{5cm} p{2cm} p{4cm} p{5.5cm}}
\hline
\textbf{Name} & \textbf{In} & \textbf{Out} & \textbf{Exceptions} \\
\hline
new SolverVisualizer & - & SolverVisualizer & - \\
\hline
setForceType & - & - & invalidConfigurationValue \\
\hline
setSolverType & - & - & invalidConfigurationValue\\
\hline
visualize & - & - & - \\
\hline
\end{tabular}
\end{center}

\subsection{Semantics}

\subsubsection{State Variables}
forceType: string\\
solverType: string
\subsubsection{Environment Variables}
None
\subsubsection{Assumptions}
None
\subsubsection{Access Routine Semantics}

\noindent \textbf{new SolverVisualizer}()
\begin{itemize}
\item transition: \\
        forceType := ""\\
		solverType := ""
\item output: out := self
\item exception: none
\end{itemize}

\noindent \textbf{setForceType}()
\begin{itemize}
\item transition: Whenever the user updates the forceType field, set the forceType string to the value of the field, and call SolverConfiguration.updateForceType(forceType).
\item exception: SolverConfiguration.updateForceType(forceType) could not parse forceType $\Rightarrow$ invalidConfigurationValue
\end{itemize}

\noindent \textbf{setSolverType}()
\begin{itemize}
\item transition: Whenever the user updates the solverType field, set the solverType string to the value of the field, and call SolverConfiguration.updateSolverType(solverType).
\item exception: SolverConfiguration.updateSolverType(solverType) could not parse solverType $\Rightarrow$ invalidConfigurationValue
\end{itemize}

\noindent \textbf{visualize}()
\begin{itemize}
\item output: Draw the input fields for forceType, solverType using the respective state variables as their values.
\item exception: none
\end{itemize}
\subsubsection{Local Functions}
None

\newpage
\section{MIS of Calculation Call Visualization} \label{CalculationCallVisualizer}

\subsection{Module}

CalculationCallVisualizer

\subsection{Uses}
\hyperref[CalculationCall]{CalculationCall} 

\subsection{Syntax}

\subsubsection{Exported Constants}
None
\subsection{Exported Types}
None

\subsubsection{Exported Access Programs}
\begin{center}
\begin{tabular}{p{6cm} p{1cm} p{5cm} p{4.5cm}}
\hline
\textbf{Name} & \textbf{In} & \textbf{Out} & \textbf{Exceptions} \\
\hline
new CalculationCallVisualizer & - & CalculationCallVisualizer & - \\
\hline
buttonPressed & - & - & invalidConfiguration \\
\hline
\end{tabular}
\end{center}

\subsection{Semantics}

\subsubsection{State Variables}
None
\subsubsection{Environment Variables}
None
\subsubsection{Assumptions}
None
\subsubsection{Access Routine Semantics}

\noindent \textbf{new CalculationCallVisualizer}()
\begin{itemize}
\item transition: none
\item output: out := self
\item exception: none
\end{itemize}

\noindent \textbf{buttonPressed}()
\begin{itemize}
\item output: When the button to calculate is pressed, call\\
CalculationCaller().runCalculation((CalculationInputT) \{ TruckConfiguration.getConfig(), BridgeConfiguration.getConfig(), SolverConfiguration.getConfig() \}).
\item exception: if TruckConfiguration(), BridgeConfiguration() are invalid configurations $\Rightarrow$ invalidConfiguration
\end{itemize}
\subsubsection{Local Functions}
None

\newpage
\section{MIS of Output Report Visualization} \label{ReportVisualizer}

\subsection{Module}

ReportVisualizer

\subsection{Uses}
\hyperref[Report]{Report}

\subsection{Syntax}

\subsubsection{Exported Constants}
None
\subsection{Exported Types}
None

\subsubsection{Exported Access Programs}
\begin{center}
\begin{tabular}{p{5cm} p{2cm} p{4cm} p{5.5cm}}
\hline
\textbf{Name} & \textbf{In} & \textbf{Out} & \textbf{Exceptions} \\
\hline
new ReportVisualizer & - & ReportVisualizer & - \\
\hline
visualize & - & - & invalidReport \\
\hline
\end{tabular}
\end{center}

\subsection{Semantics}

\subsubsection{State Variables}
None
\subsubsection{Environment Variables}
None
\subsubsection{Assumptions}
None
\subsubsection{Access Routine Semantics}

\noindent \textbf{new ReportVisualizer}()
\begin{itemize}
\item output: out := self
\item exception: none
\end{itemize}

\noindent \textbf{visualize}()
\begin{itemize}
\item output: Call Report.getReport(). To visualize the ReportT, show each field of the input alongside its output. Then, to visualize the results, show a depiction of the truck platoon and bridge above a graph. The graph should have an x-axis of the first axle positions, and the y-axis should be the force at each first axle position. For critical section analysis, there should be a second graph showing the moment envelope with bridge section on the x-axis and moment on the y-axis. Also, there should be a vertical line where the critical/concerned section is on the bridge. 
\item exception: There is no report $\Rightarrow$ invalidReport
\end{itemize}
\subsubsection{Local Functions}
None

\newpage

\bibliographystyle {plainnat}
\bibliography {../../../refs/References}

\end{document}